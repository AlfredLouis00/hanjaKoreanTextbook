\chapter{\kr 이름이 무엇입니까?}
\section{课文}
\subsection{对话1}
{\kr 오늘은 \ruby{學校}{학교}가 \ruby{始作}{시작}하는 날이다. \ruby{죤슨}{Johnson} 씨는 \ruby{敎室}{교실}로 들어갔다.\\}
{\kr
\begin{tabular}{lll}
    \ruby{朴}{박} \ruby{先生}{선생}&: & 어서 오십시오.\\
    \ruby{죤슨}{Johnson}&: &\ruby{先生}{선생}님 \ruby{安寧}{안녕}하십나까?\\
    \ruby{朴}{박} \ruby{先生}{선생}&: &앉으십시오.\\
    \ruby{죤슨}{Johnson}&: &고맙습니다.\\
\end{tabular}\\}

\noindent \textit{单词表}

\begin{tabular}{ll|ll|ll|ll|ll}
    \kr 어서&快&\kr 오다&来&\kr \ruby{先生}{선생}&老师&\kr 앉다&坐&\kr 고맙다&谢谢\\
    \color{gray} \kr 오늘\footnote{灰色字是编者加的,原书没有}&\color{gray}今天&\color{gray}\kr\ruby{學校}{학교}&\color{gray}学校&\color{gray}\kr\ruby{始作}{시작}하다&\color{gray}开始&\kr\color{gray}날&\color{gray}天,日子&\kr\color{gray}이다\footnote{本词在原书~\ref{vcb:ita}~对话单词表中首次出现}&\kr\color{gray}是\\
    \color{gray}\kr\ruby{敎室}{교실}&\color{gray}教室&\kr\color{gray} 들어가다&\color{gray}进入
\end{tabular}\\
\subsection{对话2}
{\kr
\begin{tabular}{lll}
    \ruby{朴}{박} \ruby{先生}{선생}&: &이름이 무엇입니까?\\
    \ruby{죤슨}{Johnson}&: &\ruby{톰}{Tom} \ruby{죤슨}{Johnson}입니다.\\
    \ruby{朴}{박} \ruby{先生}{선생}&: &\ruby{美國}{미국} 사람입니까?\\
    \ruby{죤슨}{Johnson}&: &예, \ruby{美國}{미국} 사람입니다.\\
\end{tabular}\\}

\noindent \textit{单词表}

\begin{tabular}{ll|ll|ll}
    \kr 이름&名字&\kr 무엇&什么&\kr 이다\label{vcb:ita}&是(用于体词后,表示谓语)\\
    \kr \ruby{美國}{미국}&美国&\kr 사람&人&\kr 예&是,对\\
\end{tabular}\\
\subsection{对话3}
{\kr
\begin{tabular}{lll}
    \ruby{朴}{박} \ruby{先生}{선생}&: &\ruby{冊}{책}이 있습니까?\\
    \ruby{죤슨}{Johnson}&: &예, 있습니다.\\
    \ruby{朴}{박} \ruby{先生}{선생}&: &\ruby{辭典}{사전}도 있습니까?\\
    \ruby{죤슨}{Johnson}&: &아니요, 없습니다.\\
\end{tabular}\\}

\noindent \textit{单词表}

\begin{tabular}{ll|ll|ll|ll|ll}
    \ruby{冊}{책}&书&있다&有&\ruby{辭典}{사전}&词典&아니요&不是&없다&没有\\
\end{tabular}\\
\subsection{短文}
\ruby{安寧}{안녕}하십나까?\\
처음 뵙겠습니다.\\
\ruby{톰}{Tom} \ruby{죤슨}{Johnson}입니다.\\
\ruby{美國}{미국}에서 왔습니다.\\
\ruby{韓國}{한국}말이 재미있습니다.\\
\ruby{先生}{선생}님이 좋습니다.\\

\noindent \textit{单词表}

\begin{tabular}{ll|ll|ll}
    처음&第一次,初次&뵙다&见面,拜会(敬语)&에서&从……(表示出发点)\\
    왔다&来了,来的&좋다&好&\ruby{韓國}{한국}말&韩国话\\
    재미있다&有趣,有意思
\end{tabular}\\
\section{\ruby{文法}{문법}}
\begin{grammar}
    \begin{grammarsect}[-ㅂ니다/-습니다.]
        \begin{itemize}
            \item 尊敬阶陈述句终结词尾,用于陈述句的结尾(参照\ref{gram:pniga}).
            \item 谓词词干末音节是开音节时用-ㅂ니다,闭音节时用-습니다.
        \end{itemize}
        \begin{tabular}{llllll}
            \ruby{例}{예}:&가다&&去,走\\
            &가&+&ㅂ니다&$\to$&갑니다\\
            &(动词词干)&&(陈述句终结词尾)\\
            &오다&&来\\
            &오&+&ㅂ니다&$\to$&옵니다\\
            &있다&&有\\
            &있&+&습니다&$\to$&있습니다\\
            &먹다&&吃\\
            &먹&+&습니다&$\to$&먹습니다\\
        \end{tabular}
    \end{grammarsect}
    \begin{grammarsect}[-ㅂ니까?/-습니까?]\label{gram:pniga}
        \begin{itemize}
            \item 尊敬阶疑问句终结词尾。
            \item 谓词词干末音节是开音节时用-ㅂ니까,闭音节时用-습니까.
        \end{itemize}
        \begin{tabular}{llllll}
            \ruby{例}{예}:&가다&&去,走\\
            &가&+&ㅂ니까&$\to$&갑니까\\
            &(动词词干)&&(疑问句终结词尾)\\
            &오다&&来\\
            &오&+&ㅂ니까&$\to$&옵니까\\
            &있다&&有\\
            &있&+&습니다&$\to$&있습니다\\
            &읽다&&读\\
            &읽&+&습니다&$\to$&읽습니다\\
        \end{tabular}
    \end{grammarsect}
    \begin{itemize}
        \item 使用-ㅂ니까/-습니까的ᅳ般疑问句,结尾时用升调,但是当句子中有疑问词 (谁、什么、什么时候、为什么、怎么) 时,结尾时用平调或降调。
    \end{itemize}
    \begin{tabular}{lll}
            \ruby{例}{예}:&\ruby{美國}{미국} 사람입니까? $\nearrow$&你是美国人吗?\\
            &\ruby{冊}{책}이 있습니까? $\nearrow$&有书吗?\\
            &이름이 무엇입니까? $\searrow$&你叫什么名字?\\
            &어디에 가십니까? $\searrow$&您去哪儿?
        \end{tabular}\\
\end{grammar}
\begin{grammar}
    \begin{grammarsect}[-이/-가]
        \begin{itemize}
            \item 助词, 用于体词后面,表示主语,有指定的意思。
            \item 体词末音节是闭音节时用-이,开音节时用-가.
        \end{itemize}
        \begin{tabular}{lll}
            \ruby{例}{예}:& 이것이 \ruby{冊}{책}입니다.&这是书。\\
            &이름이 무엇입니까?&你叫什么名字?\\
            &비가 옵니다.&下雨了。\\
            &\ruby{親舊}{친구}가 많습니다.&朋友很多。\\
            &\ruby{椅子}{의자}가 없습니다.&没有椅子。
        \end{tabular}
        \begin{itemize}
            \item 人称代词나,너,저和疑问代词누구在和助词连接时发生以下变化:
        \end{itemize}
        \begin{tabular}{llllllll}
            \ruby{例}{예}:&나 &$\to$&내&+&가&내가 말했습니다.&我说了。\\
            &저&$\to$&제&+ &가 &제가 \ruby{金英秀}{김영수}입니다. &我是金英秀。\\
            &너& $\to$ &네 &+ &가 &네가 가. &你去吧。\\
            &누구 &$\to$&누& + &가 &누가 왔습니까? &谁来了?\\
        \end{tabular}
        \begin{itemize}
            \item 当一个句子中出现两个主语时,前面的主语是整个句子的主语,后面的主语是谓语部分的主语。
        \end{itemize}
        \begin{tabular}{lll}
            \ruby{例}{예}: &\ruby{同生}{동생}이 키가 큽니다.&弟弟个子高。\\
            &그 \ruby{親舊}{친구}가 마음이 좋습니다.&那个朋友心地好。\\
            &나는 \ruby{韓國}{한국}말이 재미있습니다.&我觉得韩国话有意思。\\
            &나는 그 사람이 싫습니다.&我不喜欢那个人。\\
            &그 사람은 \ruby{親舊}{친구}가 많습니다.&那个人朋友很多。
        \end{tabular}\\
    \end{grammarsect}
    \begin{grammarsect}[-이다]
        \begin{itemize}
            \item 韩国语的谓语是由谓词(包括动词、形容词)或体词的谓词形构成。-이다用于体词后面,将体词变成体词谓词形。类似于汉语的“是-”。
        \end{itemize}
        \begin{tabular}{lll}
            \ruby{例}{예}: &이것이 \ruby{冊床}{책상}입니다.&这是书桌。\\
            &저는 \ruby{學生}{학생}입니다.&我是学生。\\
            &이 아이가 제 딸입니다.&这个孩子是我女儿。\\
            &여기가 우리집입니다.&这里是我家。\\
            &그\ruby{分}{분}은 \ruby{醫師}{의사}입니다.&他是医生。
        \end{tabular}\\
    \end{grammarsect}
    \begin{grammarsect}[-이 아니다/-가 아니다]
        \begin{itemize}
            \item -이다的否定型是-이/가 아니다。 在这样的句子中, 第ᅳ个主语是整个句子的主语,-이/가 아니다是合成谓语。在-이/가 아니다这个结构中,아니다表示否定,-이/가前面的体词是否定的对象。类似于汉语的“不是-” 。
        \end{itemize}
        \begin{tabular}{lll}
            \ruby{例}{예}:&나는 \ruby{學生}{학생}이 아닙니다.&我不是学生。\\
            &이것은 우리 집이 아닙니다.&这不是我家。\\
            &이 아이는 제 딸이 아닙니다.&这孩子不是我女儿。\\
            &여기는 \ruby{國際郵遞局}{국제우체국}이 아닙니다.&这儿不是国际邮局。\\
            &저\ruby{分}{분}은 우리 어머니가 아닙니다.&那位不是我母亲。
        \end{tabular}\\
    \end{grammarsect}
\end{grammar}
\begin{grammar}
    \begin{grammarsect}[있다]
        \begin{itemize}
            \item 있다表示拥有或存在,类似于汉语的 “有-”、“-在”。있다的否定形是없다。
        \end{itemize}
        \begin{tabular}{lll}
            \ruby{例}{예}: &\ruby{親舊}{친구}가 있습니다. &有朋友。\\
            &\ruby{時計}{시계}가 있습니다. &有手表。\\
            &그림이 없습니다. &没有图画。\\
            &\ruby{質問}{질문}이 있습니까? &有问题吗?\\
            &돈이 없습니까? &没有钱吗?
        \end{tabular}\\
    \end{grammarsect}
\end{grammar}
\begin{grammar}
    \begin{grammarsect}[이것/그것/저것]
    \begin{itemize}
        \item 이것/그것/저것是冠词이/그/저和依存名词것组合在一起构成的。冠词任何时候都置于体词的前面傲定语。
    \end{itemize}
    \begin{tabular}{lll}
        \ruby{例}{예}: &이 아이 이름이 무엇입니까?&这个孩子叫什么名字?\\
        &그 \ruby{冊}{책}은 \ruby{韓國}{한국}말 \ruby{敎科書}{교과서}입니다.&那本书是韩国语教科书。\\
        &그\ruby{分}{분}은 \ruby{事務室}{사무실}에 계십니다.&他在办公室。\\
        &저 \ruby{建物}{건물}이 우리 \ruby{學校}{학교}입니다.&那个建筑是我们学校。\\
    \end{tabular}
    \begin{itemize}
        \item 이것是指离说话者近的事物, 그것是指离听话者近而离说话者远的事物, 或者前面已经提到过的事物, 저것是指离说话者和听话者都远的事物。
    \end{itemize}
    \begin{tabular}{lll}
        \ruby{例}{예}: &이것이 무엇입니까?&这是什么?\\
        &그것이 \ruby{辭典}{사전}입니다.&那是词典。\\
        &그것은 얼마입니까?&那个多少钱?\\
        &이것이 \ruby{百}{백} 원입니다.&这个一百韩元。\\
        &저것은 \ruby{五百}{오백} 원입니다.&那个五百韩元。\\
        &오늘 \ruby{金}{김} \ruby{先生}{선생}님이 오십니다.&今天金老师来。\\
        &그것이 \ruby{事實}{사실}입니까?&那是真的吗?
    \end{tabular}\\
    \end{grammarsect}
\end{grammar}
\begin{grammar}
    \begin{grammarsect}[句子结构]
    \begin{itemize}
        \item 韩国语的句子至少由两个词构成。和其他语言ᅳ样,由主语和谓语两个部分组成。谓语放在句子的末尾,由动词、形容词或体词的谓词形构成。
    \end{itemize}
    \begin{center}
    \begin{tabular}{lllll}
        \ruby{主語}{주어}&\qquad&+&\qquad&\ruby{敘述語}{서술어}\\
        主语&\qquad&&\qquad&谓语
    \end{tabular}\\
    \end{center}
    \begin{tabular}{lll}
        \ruby{例}{예}: &아이가 웁니다.&孩子哭了。\\
        &꽃이 핍니다.&花开了。\\
        &\ruby{버스}{bus}가 많습니다.&公共汽车很多。\\
        &날씨가 좋습니다.&天气真好。\\
        &이것이 무엇입니까?&这是什么?\\
        &이\ruby{分}{분}이 \ruby{醫師}{의사}입니다.&这位是医生。
    \end{tabular}\\
    \end{grammarsect}
    \begin{itemize}
        \item 韩国语带有宾语的句子,和汉语的语序完全不同。
    \end{itemize}
    \begin{center}
    \begin{tabular}{lllll}
        \ruby{主語}{주어}&+&\ruby{目的語}{목적어}&+&\ruby{敘述語}{서술어}\\
        主语&&宾语&&谓语
    \end{tabular}\\
    \end{center}
    \begin{tabular}{lll}
        \ruby{例}{예}: &그 \ruby{學生}{학생}이 \ruby{便紙}{편지}를 씁니다.&那个学生写信。\\
        &나는 \ruby{膳物}{선물}을 삽니다.&我买礼物。\\
        &\ruby{英秀}{영수}가 \ruby{質問}{질문}을 합니다.&英秀问问题。
    \end{tabular}\\
\end{grammar}
\section{\ruby{類型}{유형} \ruby{練習}{연습}}
\begin{dic}
    \begin{dicsect}
        \begin{tabular}{rll}
            (보기) &\ruby{先生}{선생}: &김 \ruby{先生}{선생}님\\
            &\ruby{學生}{학생}: &김 \ruby{先生}{선생}님, \ruby{安寧}{안녕}하십니까?\\
            1) &\ruby{先生}{선생}: &이 \ruby{先生}{선생}님\\
            &\ruby{學生}{학생}: &이 \ruby{先生}{선생}님, \ruby{安寧}{안녕}하십니까?\\
            2) &\ruby{先生}{선생}: &박 \ruby{先生}{선생}님\\
            &\ruby{學生}{학생}: &박 \ruby{先生}{선생}님, \ruby{安寧}{안녕}하십니까?\\
            3) &\ruby{先生}{선생}: &정 \ruby{先生}{선생}님\\
            &\ruby{學生}{학생}: &정 \ruby{先生}{선생}님, \ruby{安寧}{안녕}하십니까?\\
            4) &\ruby{先生}{선생}: &\ruby{야마모토}{やまもと} \ruby{先生}{선생}님\\
            &\ruby{學生}{학생}: &야마모토 \ruby{先生}{선생}님, \ruby{安寧}{안녕}하십니까?\\
            5) &\ruby{先生}{선생}: &\ruby{스미스}{Smith} \ruby{先生}{선생}님\\
            &\ruby{學生}{학생}: &스미스 \ruby{先生}{선생}님, \ruby{安寧}{안녕}하십니까?\\
        \end{tabular}\\
    \end{dicsect}
    \begin{dicsect}
        \begin{tabular}{rll}
            (보기) &\ruby{先生}{선생}: & 하다.\\
            &\ruby{學生}{학생}: & 합니까?\\
            1) &\ruby{先生}{선생}: & 가다.\\
            &\ruby{學生}{학생}: & 갑니까?\\
            2) &\ruby{先生}{선생}: & 오다.\\
            &\ruby{學生}{학생}: & 옵니까?\\
            3) &\ruby{先生}{선생}: & \ruby{功夫}{공부}하다.\\
            &\ruby{學生}{학생}: & \ruby{功夫}{공부}합니까?\\
            4) &\ruby{先生}{선생}: & 읽다.\\
            &\ruby{學生}{학생}: & 읽습니까?\\
            5) &\ruby{先生}{선생}: & 찾다.\\
            &\ruby{學生}{학생}: & 찾습니까?
        \end{tabular}\\
    \end{dicsect}
    \begin{dicsect}
        \begin{tabular}{rll}
            (보기) &\ruby{先生}{선생}: & 합니까? (예)\\
            &\ruby{學生}{선생}: & 예,합니다.\\
            1) &\ruby{先生}{선생}: &갑니까? (예)\\
            &\ruby{學生}{학생}: & 예,갑니다. \\
            2) &\ruby{先生}{선생}: &옵니까? (예)\\ 
            &\ruby{學生}{학생}: & 예,옵니다. \\
            3) &\ruby{先生}{선생}: &\ruby{功夫}{공부}합니까? (예)\\ 
            &\ruby{學生}{학생}: & 예, \ruby{功夫}{공부}합니다.\\
            4) &\ruby{先生}{선생}: &읽습니까? (예)\\ 
            &\ruby{學生}{학생}: & 예, 읽습니다.\\
            5) &\ruby{先生}{선생}: &찾습니까? (예)\\ 
            &\ruby{學生}{학생}: & 예, 찾습니다.
        \end{tabular}\\
    \end{dicsect}
\end{dic}
\begin{dic}
    \begin{dicsect}
        \begin{tabular}{rlll}
            (보기) &\ruby{先生}{선생}: & 무엇\\
            &\ruby{學生}{학생}: & 무엇입니까?\\
            1) &\ruby{先生}{선생}: & \ruby{스미스}{Smith} \ruby{先生}{선생}님\\
            &\ruby{學生}{학생}: & \ruby{스미스}{Smith} \ruby{先生}{선생}님입니까?\\
            2) &\ruby{先生}{선생}: & \ruby{韓國}{한국} 사람\\
            &\ruby{學生}{학생}: & \ruby{韓國}{한국} 사람입니까?\\
            3) &\ruby{先生}{선생}: & \ruby{中國}{중국} 사람\\
            &\ruby{學生}{학생}: & \ruby{中國}{중국} 사람입니까?\\
            4) &\ruby{先生}{선생}: & \ruby{學生}{학생} \\
            &\ruby{學生}{학생}: & \ruby{學生}{학생}입니까? \\
            5) &\ruby{先生}{선생}: & \ruby{親舊}{친구} \\
            &\ruby{學生}{학생}: & \ruby{親舊}{친구}입니까?
        \end{tabular}\\
    \end{dicsect}
    \begin{dicsect}
        \begin{tabular}{rll}
            (보기) &\ruby{先生}{선생}: & \ruby{스미스}{Smith} \ruby{先生}{선생}님입니까? (예) \\
            &\ruby{學生}{학생}: & 예,\ruby{스미스}{Smith}입니다.\\
            1) &\ruby{先生}{선생}: & \ruby{韓國}{한국} 사람입니까? (예) \\
            &\ruby{學生}{학생}: & 예, \ruby{韓國}{한국} 사람입니다.\\
            2) &\ruby{先生}{선생}: & \ruby{中國}{중국} 사람입니까? (예)\\ 
            &\ruby{學生}{학생}: & 예, \ruby{中國}{중국} 사람입니다.\\
            3) &\ruby{先生}{선생}: & \ruby{學生}{학생}입니까? (예) \\
            &\ruby{學生}{학생}: & 예,\ruby{學生}{학생}입니다. \\
            4) &\ruby{先生}{선생}: & \ruby{親舊}{친구}입니까? (예) \\
            &\ruby{學生}{학생}: & 예, \ruby{親舊}{친구}입니다.\\
            5) &\ruby{先生}{선생}: & \ruby{敎科書}{교과서}입니까? (예) \\
            &\ruby{學生}{학생}: & 예,\ruby{敎科書}{교과서}입니다.
        \end{tabular}\\
    \end{dicsect}
    \begin{dicsect}
        \begin{tabular}{rll}
            (보기) &\ruby{先生}{선생}: & \ruby{韓國}{한국} 사람입니까? (아니요/\ruby{中國}{중국}사람) \\
            &\ruby{學生}{학생}: & 아니요,\ruby{中國}{중국} 사람입니다.\\
            1) &\ruby{先生}{선생}: & \ruby{美國}{미국} 사람입니까? (아니요/\ruby{英國}{영국}사람) \\
            &\ruby{學生}{학생}: & 아니요,\ruby{英國}{영국} 사람입니다.\\
            2) &\ruby{先生}{선생}: & \ruby{先生}{선생}님입니까? (아니요/\ruby{學生}{학생}) \\
            &\ruby{學生}{학생}: & 아니요,\ruby{學生}{학생}입니다. \\
            3) &\ruby{先生}{선생}: & \ruby{스미스}{Smith} \ruby{先生}{선생}님입니까? (아니요/\ruby{죤슨}{Johnson}) \\
            &\ruby{學生}{학생}: & 아니요,\ruby{죤슨}{Johnson}입니다. \\
            4) &\ruby{先生}{선생}: & \ruby{敎科書}{교과서}입니까? (아니요/\ruby{雜誌}{잡지}) \\
            &\ruby{學生}{학생}: & 아니요,\ruby{雜誌}{잡지}입니다. \\
            5) &\ruby{先生}{선생}: & \ruby{鉛筆}{연필}입니까? (아니요/\ruby{볼}{ball}\ruby{펜}{pen}) \\
            &\ruby{學生}{학생}: & 아니요,\ruby{볼}{ball}\ruby{펜}{pen}입니다. 
        \end{tabular}\\
    \end{dicsect}
    \begin{dicsect}
        \begin{tabular}{rll}
            (보기) &\ruby{先生}{선생}: & \ruby{韓國}{한국} 사람입니까? (아니요) \\
            &\ruby{學生}{학생}: & 아니요, \ruby{韓國}{한국} 사람이 아닙니다.\\
            1) &\ruby{先生}{선생}: & \ruby{美國}{미국} 사람입니까? (아니요) \\
            &\ruby{學生}{학생}: & 아니요,\ruby{美國}{미국} 사람이 아닙니다.\\
            2) &\ruby{先生}{선생}: & \ruby{親舊}{친구}입니까? (아니요) \\
            &\ruby{學生}{학생}: & 아니요,\ruby{親舊}{친구}가 아닙니다.\\
            3) &\ruby{先生}{선생}: & \ruby{鉛筆}{연필}입니까? (아니요)\\ 
            &\ruby{學生}{학생}: & 아니요,\ruby{鉛筆}연필{}이 아닙니다.\\
            4) &\ruby{先生}{선생}: & \ruby{敎科書}{교과서}입니까? (아니요)\\ 
            &\ruby{學生}{학생}: & 아니요,\ruby{敎科書}{교과서}가 아닙니다.\\
            5) &\ruby{先生}{선생}: & \ruby{雜誌}{잡지}입니까? (아니요) \\
            &\ruby{學生}{학생}: & 아니요,\ruby{雜誌}{잡지}가 아닙니다.
        \end{tabular}\\
    \end{dicsect}
\end{dic}

\begin{dic}
    \begin{dicsect}
        \begin{tabular}{rll}
            (보기) &\ruby{先生}{선생}: & \ruby{敎科書}{교과서}\\
            &\ruby{學生}{학생}: & \ruby{敎科書}{교과서}가 있습니까?\\
            1) &\ruby{先生}{선생}: & \ruby{鉛筆}{연필}\\
            &\ruby{學生}{학생}: & \ruby{鉛筆}{연필}이있습니까? \\
            2) &\ruby{先生}{선생}: & \ruby{親舊}{친구}\\
            &\ruby{學生}{학생}: & \ruby{親舊}{친구}가있습니까?\\
            3) &\ruby{先生}{선생}: & 돈\\
            &\ruby{學生}{학생}: & 돈이 있습니까?\\
            4) &\ruby{先生}{선생}: & \ruby{時計}{시계} \\
            &\ruby{學生}{학생}: & \ruby{時計}{시계}가 있습니까?\\
            5) &\ruby{先生}{선생}: & \ruby{가방}{かばん} \\
            &\ruby{學生}{학생}: & \ruby{가방}{かばん}이 있습니까?
        \end{tabular}\\
    \end{dicsect}
    \begin{dicsect}
        \begin{tabular}{rll}
            (보기) &\ruby{先生}{선생}: & \ruby{鉛筆}{연필}이 있습니까? (예) \\
            &\ruby{學生}{학생}: & 예,\ruby{鉛筆}{연필}이있습니다.\\
            1) &\ruby{先生}{선생}: & \ruby{親舊}{친구}가 있습니까? (예) \\
            &\ruby{學生}{학생}: & 예, \ruby{親舊}{친구}가 있습니다.\\
            2)&\ruby{先生}{선생}: & 돈이 있습니까? (예)\\
            &\ruby{學生}{학생}: & 예, 돈이 있습니다.\\
            3) &\ruby{先生}{선생}: & \ruby{時計}{시계}가 있습니까? (예) \\
            &\ruby{學生}{학생}: & 예, \ruby{時計}{시계}가 있습니다.\\
            4) &\ruby{先生}{선생}: & \ruby{時間}{시간}이 있습니까? (예) \\
            &\ruby{學生}{학생}: & 예, \ruby{時間}{시간}이 있습니다.\\
            5) &\ruby{先生}{선생}: & \ruby{質問}{질문}이 있습니까? (예) \\
            &\ruby{學生}{학생}: & 예,\ruby{質問}{질문}이 있습니다.
        \end{tabular}\\
    \end{dicsect}
    \begin{dicsect}
        \begin{tabular}{rll}
            (보기) &\ruby{先生}{선생}: & \ruby{辭典}{사전}이 있습니까? (아니요) \\
            &\ruby{學生}{학생}: & 아니요,\ruby{辭典}{사전}이 없습니다.\\
            1) &\ruby{先生}{선생}: & \ruby{質問}{질문}이 있습니까? (아니요) \\
            &\ruby{學生}{학생}: & 아니요, \ruby{質問}{질문}이 없습니다.\\
            2) &\ruby{先生}{선생}: & \ruby{時間}{시간}이 있습니까? (아니요) \\
            &\ruby{學生}{학생}: & 아니요,\ruby{時間}{시간}이 없습니다.\\
            3) &\ruby{先生}{선생}: & \ruby{冊床}{책상}이 있습니까? (아니요) \\
            &\ruby{學生}{학생}: & 아니요,\ruby{冊床}{책상}이 없습니다.\\
            4) &\ruby{先生}{선생}: & \ruby{椅子}{의자}가 있습니까? (아니요) \\
            &\ruby{學生}{학생}: & 아니요,\ruby{椅子}{의자}가 없습니다.\\
            5) &\ruby{先生}{선생}: & \ruby{地圖}{지도}가 있습니까? (아니요) \\
            &\ruby{學生}{학생}: & 아니요,\ruby{地圖}{지도}가 없습니다.
        \end{tabular}\\
    \end{dicsect}
    \begin{dicsect}
        \begin{tabular}{rll}
            (보기) &\ruby{先生}{선생}: & \ruby{先生}{선생}님\\
            &\ruby{學生}{학생}: & \ruby{先生}{선생}님이 계십니까?\\
            1) &\ruby{先生}{선생}: & 어머니 \\
            &\ruby{學生}{학생}: & 어머니가 계십니까?\\
            2) &\ruby{先生}{선생}: & 아버지 \\
            &\ruby{學生}{학생}: & 아버지가 계십니까?\\
            3) &\ruby{先生}{선생}: & \ruby{醫師}{의사} \ruby{先生}{선생}님\\
            &\ruby{學生}{학생}: & \ruby{醫師}{의사} \ruby{先生}{선생}님이 계십니까?\\
            4) &\ruby{先生}{선생}: & \ruby{主人}{주인} \\
            &\ruby{學生}{학생}: & \ruby{主人}{주인}이 계십니까?\\
            5) &\ruby{先生}{선생}: & 김 \ruby{博士}{박사}님\\
            &\ruby{學生}{학생}: & 김 \ruby{博士}{박사}님이 계십니까?\\
        \end{tabular}\\
    \end{dicsect}
\end{dic}
\begin{dic}
    \begin{dicsect}
        \begin{tabular}{rll}
            (보기) &\ruby{先生}{선생}: & 그것/\ruby{辭典}{사전}\\
            &\ruby{學生}{학생}: & 그것이 \ruby{辭典}{사전}입니까?\\
            1) &\ruby{先生}{선생}: & 이것/\ruby{宿題}{숙제}\\
            &\ruby{學生}{학생}: & 이것이 \ruby{宿題}{숙제}입니까?\\
            2) &\ruby{先生}{선생}: & 저것/\ruby{地圖}{지도}\\
            &\ruby{學生}{학생}: & 저것이 \ruby{地圖}{지도}입니까?\\
            3) &\ruby{先生}{선생}: & 그것/\ruby{新聞}{신문}\\
            &\ruby{學生}{학생}: & 그것이 \ruby{新聞}{신문}입니까?\\
            4) &\ruby{先生}{선생}: & 저것/\ruby{南大門}{남대문}\\
            &\ruby{學生}{학생}: & 저것이 \ruby{南大門}{남대문}입니까?\\
            5) &\ruby{先生}{선생}: & 이것/불고기\\
            &\ruby{學生}{학생}: & 이것이 불고기입니까?
        \end{tabular}\\
    \end{dicsect}
    \begin{dicsect}
        \begin{tabular}{rll}
            (보기) &\ruby{先生}{선생}: & 그것이 \ruby{地圖}{지도}입니까? (예/이것)\\
            &\ruby{學生}{학생}: & 예, 이것이 \ruby{地圖}{지도}입니다.\\
            1) &\ruby{先生}{선생}: & 그것이 \ruby{新聞}{신문}입니까? (예/이것)\\
            &\ruby{學生}{학생}: & 예, 이것이 \ruby{新聞}{신문}입니다.\\
            2) &\ruby{先生}{선생}: & 이것이 \ruby{英語}{영어} \ruby{敎科書}{교과서}입니까? (예/그것)\\
            &\ruby{學生}{학생}: & 예, 그것이 \ruby{英語}{영어} \ruby{敎科書}{교과서}입니다.\\
            3) &\ruby{先生}{선생}: & 이것이 김치입니까? (예/이것)\\
            &\ruby{學生}{학생}: & 예, 이것이 김치입니다.\\
            4) &\ruby{先生}{선생}: & 저것이 \ruby{門}{문}입니까? (예/저것)\\
            &\ruby{學生}{학생}: & 예,저것이 \ruby{門}{문}입니다.\\
            5) &\ruby{先生}{선생}: & 그것이 \ruby{커피}{coffee}입니까? (예/이것)\\
            &\ruby{學生}{학생}: & 예,이것이 \ruby{커피}{coffee}입니다.\\
        \end{tabular}\\
    \end{dicsect}
    \begin{dicsect}
        \begin{tabular}{rll}
            (보기) &\ruby{先生}{선생}: & 그것이 \ruby{地圖}{지도}입니까? (아니요/\ruby{新聞}{신문}) \\
            &\ruby{學生}{학생}: & 아니요,\ruby{新聞}{신문}입니다.\\
            1) &\ruby{先生}{선생}: & 그것이 \ruby{新聞}{신문}입니까? (아니요/\ruby{雜誌}{잡지}) \\
            &\ruby{學生}{학생}: & 아니요,\ruby{雜誌}{잡지}입니다.\\
            2) &\ruby{先生}{선생}: & 저것이 \ruby{門}{문}입니까? (아니요/\ruby{窓門}{창문}) \\
            &\ruby{學生}{학생}: & 아니요,\ruby{窓門}{창문}입니다.\\
            3) &\ruby{先生}{선생}: & 저것이 \ruby{南大門}{남대문}입니까? (아니요/\ruby{東大門}{동대문}) \\
            &\ruby{學生}{학생}: & 아니요,\ruby{東大門}{동대문}입니다.\\
            4) &\ruby{先生}{선생}: & 이것이 \ruby{커피}{coffee}입니까? (아니요/보리\ruby{茶}{차}) \\
            &\ruby{學生}{학생}: & 아니요,보리\ruby{茶}{차}입니다.\\
            5) &\ruby{先生}{선생}: & 그것이 \ruby{볼}{ball}\ruby{펜}{pen}입니까? (아니요/\ruby{鉛筆}{연필}) \\
            &\ruby{學生}{학생}: & 아니요,\ruby{鉛筆}{연필}입니다.
        \end{tabular}\\
    \end{dicsect}
    \begin{dicsect}
        \begin{tabular}{rll}
            (보기) &\ruby{先生}{선생}: & \ruby{學生}{학생}이 많습니까? (예) \\
            &\ruby{學生}{학생}: & 예,\ruby{學生}{학생}이 많습니다.\\
            1) &\ruby{先生}{선생}: & 돈이 많습니까? (예) \\
            &\ruby{學生}{학생}: &예, 돈이 많습니다.\\
            2)&\ruby{先生}{선생}: &\ruby{親舊}{친구}가많습니까? (예) \\
            &\ruby{學生}{학생}: &예,\ruby{親舊}{친구}가 많습니다.\\
            3)&\ruby{先生}{선생}: &\ruby{時間}{시간}이 많습니까? (예)\\ 
            &\ruby{學生}{학생}: &예,\ruby{時間}{시간}이 많습니다.\\
            4)&\ruby{先生}{선생}: &사람이 많습니까? (예)\\ 
            &\ruby{學生}{학생}: &예,사람이 많습니다.\\
            5)&\ruby{先生}{선생}: &\ruby{宿題}{숙제}가 많습니까? (예)\\ 
            &\ruby{學生}{학생}: &예,\ruby{宿題}{숙제}가 많습니다.
        \end{tabular}\\
    \end{dicsect}
\end{dic}
\begin{dic}
    \begin{dicsect}
        \begin{tabular}{rll}
            (보기) &\ruby{先生}{선생}: & \ruby{美國}{미국}\\
            &\ruby{學生}{학생}1: & \ruby{美國}{미국}에서 왔습니까? (예) \\
            &\ruby{學生}{학생}2: & 예,\ruby{美國}{미국}에서 왔습니다.\\
            1) &\ruby{先生}{선생}: &\ruby{中國}{중국} \\
            &\ruby{學生}{학생}1: &\ruby{中國}{중국}에서 왔습니까? (예) \\
            &\ruby{學生}{학생}2: &예,\ruby{中國}{중국}에서 왔습니다.\\
            2) &\ruby{先生}{선생}: &\ruby{英國}{영국} \\
            &\ruby{學生}{학생}1: &\ruby{英國}{영국}에서 왔습니까? (예) \\
            &\ruby{學生}{학생}2: &예,\ruby{英國}{영국}에서 왔습니다.\\
            3) &\ruby{先生}{선생}: &\ruby{日本}{일본} \\
            &\ruby{學生}{학생}1: &\ruby{日本}{일본}에서 왔습니까? (예) \\
            &\ruby{學生}{학생}2: &예,\ruby{日本}{일본}에서 왔습니다.\\
        \end{tabular}\\
        \begin{tabular}{rll}
            {\color{white} 1111}4) &\ruby{先生}{선생}: &\ruby{獨逸}{독일} \\
            &\ruby{學生}{학생}1: &\ruby{獨逸}{독일}에서 왔습니까? (예) \\
            &\ruby{學生}{학생}2: &예,\ruby{獨逸}{독일}에서 왔습니다.\\
            5) &\ruby{先生}{선생}: & \ruby{러시아}{Russia}\\
            &\ruby{學生}{학생}1: & \ruby{러시아}{Russia}에서 왔습니까? (예) \\
            &\ruby{學生}{학생}2: & 예,\ruby{러시아}{Russia}에서 왔습니다.
        \end{tabular}\\
    \end{dicsect}
    \begin{dicsect}
        \begin{tabular}{rll}
            (보기) &\ruby{先生}{선생}: & 어디에서 왔습니까? (\ruby{美國}{미국}) \\
            &\ruby{學生}{학생}: & \ruby{美國}{미국}에서 왔습니다.\\
            1) &\ruby{先生}{선생}: &어디에서 왔습니까? (\ruby{獨逸}{독일}) \\
            &\ruby{學生}{학생}: &\ruby{獨逸}{독일}에서 왔습니다.\\
            2) &\ruby{先生}{선생}: &어디에서 왔습니까? (\ruby{프랑스}{France}) \\
            &\ruby{學生}{학생}: &\ruby{프랑스}{France}에서 왔습니다.\\
            3) &\ruby{先生}{선생}: &어디에서 왔습니까? (\ruby{캐나다}{Canada}) \\
            &\ruby{學生}{학생}: &\ruby{캐나다}{Canada}에서 왔습니다.\\
            4) &\ruby{先生}{선생}: &어디에서 왔습니까? (\ruby{釜山}{부산}) \\
            &\ruby{學生}{학생}: &\ruby{釜山}{부산}에서 왔습니다.\\
            5) &\ruby{先生}{선생}: &어디에서 왔습니까? (\ruby{大邱}{대구}) \\
            &\ruby{學生}{학생}: &\ruby{大邱}{대구}에서 왔습니다.\\
        \end{tabular}\\
    \end{dicsect}
    \begin{dicsect}
        \begin{tabular}{rll}
            (보기) &\ruby{先生}{선생}: & \ruby{韓國}{한국}말/재미있습니다.\\
            &\ruby{學生}{학생}: & \ruby{韓國}{한국}말이 재미있습니다.\\
            1) &\ruby{先生}{선생}: & \ruby{時計}{시계}/좋습니다.\\
            &\ruby{學生}{학생}: & \ruby{時計}{시계}가 좋습니다.\\
            2)&\ruby{先生}{선생}: &\ruby{敎科書}{교과서}/많습니다.\\
            &\ruby{學生}{학생}: &\ruby{敎科書}{교과서}가 많습니다.\\
            3) &\ruby{先生}{선생}: &\ruby{學生}{학생}/\ruby{功夫}{공부}합니다. \\
            &\ruby{學生}{학생}: &\ruby{學生}{학생}이 \ruby{功夫}{공부}합니다. \\
            4) &\ruby{先生}{선생}: &\ruby{親舊}{친구}/갑니다. \\
            &\ruby{學生}{학생}: &\ruby{親舊}{친구}가 갑니다. \\
            5) &\ruby{先生}{선생}: &이것/\ruby{宿題}{숙제}입니다. \\
            &\ruby{學生}{학생}: &이것이 \ruby{宿題}{숙제}입니다.
        \end{tabular}\\
    \end{dicsect}
\end{dic}