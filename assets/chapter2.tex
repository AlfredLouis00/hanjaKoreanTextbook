\chapter{\kr \ruby{工夫}{공부}하기가 재미있습니다.}
\section{课文}
\subsection{对话}
{\kr \ruby{죤슨}{Johnson} 씨는 \ruby{敎室}{교실}에서 나와서 \ruby{金美善}{김미선} 씨를 만났다.\\}

{\kr
\begin{tabular}{lll}
    \ruby{죤슨}{Johnson} &:&\ruby{食堂}{식당}이 어디에 있습니까?\\
    \ruby{金美善}{김미선} &:& 저기에 있습니다.\\
    \ruby{죤슨}{Johnson} &:& \ruby{圖書館}{도서관}은 어디에 있습니까?\\
    \ruby{金美善}{김미선} &:&저 뒤에 있습니다. 같이 갑시다.\\
    \ruby{죤슨}{Johnson} &:& \ruby{未安}{미안}합니다.\\ 
    \ruby{金美善}{김미선} &:& 괜찮습니다.\\
\end{tabular}\\}

\noindent \textit{单词表}\\

\begin{tabular}{ll|ll|ll}
    \kr \ruby{食堂}{식당}&食堂,饭店&\kr 어디&什么地方,哪儿&\kr 저기&那儿(远称)\\
    \kr \ruby{圖書館}{도서관}&图书馆&\kr 저 뒤에&在那后面&\kr 같이&一起\\
    \kr 괜찮다\footnote{来自{\kr 괜하지아니하다},其中{\kr 괜}来自汉字词\kr 空然(공연)}&没关系
\end{tabular}\\
\subsection{对话}
{\kr 두 사람은 \ruby{圖書館}{도서관} 쪽으로간다.\\

\begin{tabular}{lll}
    \ruby{金美善}{김미선} &:& \ruby{工夫}{공부}하기가 어떻습니까?\\
    \ruby{죤슨}{Johnson} &:& 재미있습니다.\\
    \ruby{美善}{미선} &:& 누가 가르치십니까?\\
    \ruby{죤슨}{Johnson} &:& \ruby{朴}{박} \ruby{先生}{선생}님이 가르치십니다.\\
    \ruby{美善}{미선} &:& \ruby{學生}{학생}들이 \ruby{熱心}{열심}히 \ruby{工夫}{공부}합니까?\\
    \ruby{죤슨}{Johnson} &:& 예,\ruby{熱心}{열심}히 합니다.\\
\end{tabular}\\}

\noindent \textit{单词表} \\

\begin{tabular}{ll|ll|ll}
    \kr \ruby{工夫}{공부}하다 &学习 &\kr 어떻다& 怎么样 &\kr 누구 &谁\\
    \kr 가르치다 &教 &\kr\ruby{學生}{학생} &学生 &-들 &-们,-些\\
    \kr \ruby{熱心}{열심}히 &用功地,认真地&&&&(表示复数) 
\end{tabular}\\

\subsection{对话}
{\kr 두 사람이 가는데 어떤 \ruby{學生}{학생}이 \ruby{金美善}{김미선} 씨에게 \ruby{人事}{인사}를 하고 지나갔다.\\

\begin{tabular}{lll}
    \ruby{죤슨}{Johnson} &:& 그분이 누구입니까?\\
    \ruby{美善}{미선}&:& \ruby{親舊}{친구}입니다.\\
    \ruby{죤슨}{Johnson} &:& 무엇을 \ruby{工夫}{공부}합니까?\\
    \ruby{美善}{미선}&:& \ruby{歷史}{역사}를 \ruby{工夫}{공부}합니다.\\
    \ruby{죤슨}{Johnson} &:& 저도 \ruby{歷史}{역사}를 배우고 싶습니다.\\
    \ruby{美善}{미선}&:& 그렇습니까?
\end{tabular}\\
}

\noindent \textit{单词表}\\

\begin{tabular}{ll|ll|ll}
    \kr 그분 &那位,他 &\kr \ruby{親舊}{친구} &朋友 &\ruby{歷史}{역사} &历史\\
    \kr 저 &我 (谦称)  &\kr -도 &……也 &\kr 배우다 &学习\\
    \kr -고 싶다. &想,希望 &\kr 그렇다 &那样,是那样
\end{tabular}\\

\subsection{对话}
{\kr 두 사람은 \ruby{圖書館}{도서관}으로 들어갔다.\\

\begin{tabular}{lll}
    \ruby{죤슨}{Johnson} &:& 이것은 무슨 \ruby{冊}{책}입니까?\\
    \ruby{美善}{미선} &:& \ruby{小說冊}{소설책}입니다.\\
    \ruby{죤슨}{Johnson} &:& \ruby{新聞}{신문}하고 \ruby{雜誌}{잡지}는 어디에 있습니까?\\
    \ruby{美善}{미선} &:& 저 \ruby{房}{방}에 있습니다.\\
    \ruby{죤슨}{Johnson} &:& 이제 나갑시다.\\
    \ruby{美善}{미선} &:& 좋습니다. 나갑시다.\\
\end{tabular}\\}

\noindent \textit{单词表}\\

\begin{tabular}{ll|ll|ll}
    \kr 무슨 &什么 &\kr\ruby{小說冊}{소설책} &小说(书) &\ruby{新聞}{신문} &报纸\\
    \kr \ruby{雜誌}{잡지} &杂志 &\ruby{房}{방} &屋子,房间 &이제 &现在\\
    \kr 나가다 &出去
\end{tabular}\\
\subsection{短文}
{\kr 나는 날마다 \ruby{圖書館}{도서관}에 갑니다.\\
\ruby{圖書館}{도서관}에는 \ruby{冊}{책}이 많습니다.\\
\ruby{學生}{학생}도 많습니다.\\
\ruby{學生}{학생}들이 \ruby{熱心}{열심}히 \ruby{工夫}{공부}합니다.\\
나는 \ruby{冊}{책}을 읽습니다.\\
\ruby{雜誌}{잡지}하고 \ruby{新聞}{신문}도 읽습니다.\\
\ruby{宿題}{숙제}도 합니다.\\}

\noindent \textit{单词表}\\

\begin{tabular}{ll|ll|ll}
    나 &我 &날마다 &每天 &읽다 &读
\end{tabular}\\
\section{\kr\ruby{文法}{문법}}
\begin{grammar}
    \begin{grammarsect}[\kr -에]
        \begin{itemize}
            \item 助词。用于表示地点的体词后,表示主体的存在处所,或趋向动词的目的地。
        \end{itemize}
        \begin{tabular}{lll}
            \kr\ruby{例}{예}:& \kr\ruby{英秀}{영수}는 집에 있습니다.&英秀在家。\\
            &\kr\ruby{父母}{부모}님이 \ruby{故鄉}{고향}에 계십니다.&父母在老家。\\
            &\kr\ruby{市場}{시장}에 \ruby{物件}{물건}이 많습니다.&市场里有很多东西。\\
            &\kr 날마다 \ruby{圖書館}{도서관}에 갑니다.&每天去图书馆。\\
            &\kr 어디에 가십니까?&您去哪儿?\\
            &\kr\ruby{來日}{내일} 우리 집에 오십시오.&明天来我家吧。
        \end{tabular}\\
        \begin{itemize}
            \item 表示存在处所的助词{\kr-에}加上动词{\kr 있다},构成{\kr -에 있다}的句型,表示$\times \times$在$\times \times$ (某处)。
        \end{itemize}
        \begin{tabular}{lll}
            \kr \ruby{例}{예}:&\kr \ruby{冊}{책}이 \ruby{冊床}{책상}에 있습니다.&书在桌子上。\\ 
            &\kr \ruby{圖書館}{도서관}이 어디에 있습니까?&图书馆在哪儿?\\
            &\kr \ruby{同生}{동생}이 \ruby{美國}{미국}에 있습니다.&弟弟在美国。\\
            &\kr \ruby{學生}{학생}들이 \ruby{教室}{교실}에 없습니다.&学生们不在教室\\
            &\kr 어머니는 집에 계십니다.&妈妈在家。\\
        \end{tabular}\\
    \end{grammarsect}
    \begin{grammarsect}[\kr 여기, 거기, 저기]
        \begin{itemize}
            \item 处所代词.离说话者近的地方用{\kr 여기},离听者近而离说话者远的地方,或者前面己经提过的地方用{\kr 거기},离说话者和听话者都远的地方用{\kr 저기}.
        \end{itemize}
        \begin{tabular}{lll}
            \kr \ruby{例}{예}:&\kr  여기가 어디입니까?&这是什么地方? \\
            &\kr 여기에 앉으십시오.&请坐这儿吧。\\
            &\kr 거기에 무엇이 있습니까?&那儿有什么?\\
            &\kr \ruby{來日}{내일} 거기에서 만납시다.&明天在那儿见面吧。\\
            &\kr 저기가 \ruby{南山}{남산}입니다.&那儿是南山。
        \end{tabular}\\
    \end{grammarsect}
    \begin{grammarsect}[尊敬阶终结词尾]\label{gram:ref}
        \begin{itemize}
            \item 尊敬阶终结词尾用于句子的结尾,置于谓词词干后。
        \end{itemize}
        \begin{tabular}{llcccc}
            \kr 나는 갑니다.&$\to $&\kr 나&\kr 는&\kr 가&\kr ㅂ니다\\
            &&名词&补助词&动词词干&终结词尾\\\cmidrule(r){3-4} \cmidrule(l){5-6}
            &&\multicolumn{2}{c}{主语}&\multicolumn{2}{c}{谓语}
        \end{tabular}\\
        \begin{itemize}
            \item 根据终结词尾的不同,韩国语的句子可以分为四种类型。
        \end{itemize}
        \begin{tabular}{|c|c|c|c|c|}\hline
            \diagbox{谓词类型}{句子类型}&陈述句&疑问句&命令句&共动句\\\hline
            以开音节结尾的谓词 &\kr -ㅂ니다 &\kr -ㅂ니까 &\kr -십시오 &\kr -ㅂ시다\\\hline
            以闭音节结尾的谓词 &\kr -습니다 &\kr -습니까 &\kr -으십시오 &\kr -옵시다\\\hline
        \end{tabular}\\
        \begin{itemize}
            \item 表示尊敬的终结词尾有尊敬阶终结词尾和准敬阶终结词尾, 尊敬阶终结词尾用于正式场合,准敬阶终结词尾在表示尊敬的同时,带有亲切柔和的语气,多用于关系亲密的人之间的私人对话。
        \end{itemize}
    \end{grammarsect}
    \begin{grammarsect}[\kr -ㅂ시오]
        \begin{itemize}
            \item 尊敬阶命令式终结词尾,表示说话人请听话人做某种行为。和尊称词尾{\kr -으시/시-}共同组成 (参照\ref{gram:si}) 。动词词干末音节是开音节时用-십시오;
闭音节时用{\kr -으십시오}。 
            {\color{gray} \item Ⅱ*{\kr -십시오}} 
        \end{itemize}
        \begin{tabular}{ccccccccc}
            \kr \ruby{例}{예}:&\kr 가다&&&&去,走\\
            &\kr 가&+&\kr 시&+&\kr ㅂ시오&$\to$&\kr 가십시오.\\\cline{2-2}\cline{4-4}\cline{6-6}
            &动词词干&&尊称词尾&&终结词尾
        \end{tabular}\\
        \begin{tabular}{lcccc}
            \kr 오다 &&&&来\\
            \kr 오 &+ &\kr 십시오&$\to$&\kr 오십시오\\
            \kr 읽다 &&&&读\\
            \kr 읽 &+&\kr 으십시오&$\to$&\kr 읽으십시오\\
            \kr 찾다 &&&&找 \\
            \kr 찾 &+ &\kr 으십시오&$\to$&\kr 찾으십시오 
        \end{tabular}\\
    \end{grammarsect}
    \begin{grammarsect}[\kr -(으){}ㅂ시다]
        \begin{itemize}
            \item 尊敬阶共动式终结词尾。说话人要求对方和自己一起去傲某个行动。动词词干末音节是开音节时用{\kr -ㅂ시다}, 闭音节时用{\kr -읍시다}。
            {\color{gray} \item Ⅱ*{\kr -십시다}} 
        \end{itemize}
        \begin{tabular}{ccccccc}
            \kr \ruby{例}{예}:&\kr 가다&&&&去,走\\
            &\kr 가&+&\kr ㅂ시다&$\to$\footnote{原书这里是用的``+''但是考虑内容应该是``$\to$''}&\kr 갑시다\\\cline{2-2}\cline{4-4}
            &动词词干&&尊敬阶共动式终结词尾\\
            &\kr 읽다 &&&&读\\
            &\kr 읽 &+&\kr 읍시다&$\to$&\kr  읽읍시다\\
            &\kr 앉다 &&&&坐\\
            &\kr 앉 &+&\kr 읍시다&$\to$&\kr 앉읍시다
        \end{tabular}\\
    \end{grammarsect}
\end{grammar}
\begin{grammar}
    \begin{grammarsect}[\kr -(으){}시-]\label{gram:si}
        \begin{itemize}
            \item 尊称词尾。用于谓词词干和体词谓词形后.谓词词干末音节是开音节时用{\kr -시-},闭音节时用{\kr -으시-},体词谓词形后用{\kr -시-}。表示对动作、状态、性质主体的尊敬。
            {\color{gray} \item Ⅱ*{\kr -시-},此处的{\kr -시-}是中缀,也有三类语基,变化规则,分别为{\kr 시-,시-,셔-}}
        \end{itemize}
        \begin{tabular}{lll}
            \kr \ruby{例}{예}: &\kr \ruby{先生}{선생}님이 오십니다.&老师来了。\\
            &\kr 이분은 \ruby{夫人}{부인}이 십니까?&这位是夫人吗?\\ 
            &\kr \ruby{來日}{내일} \ruby{會社}{회사}에 가시겠습니까?&您明天去公司吗?\\
            &\kr 아버지는 \ruby{新聞}{신문}을 읽으십니다.&爸爸在读报纸。\\
            &\kr \ruby{先生}{선생}님은 \ruby{工夫}{공부}하기가 재미 있으십니까?&先生\footnote{这里作“老师”更符合汉语习惯}您学习有意思吗?
        \end{tabular}\\
        \begin{itemize}
            \item 有些词不能用尊称词尾{\kr -시-},而必须换成自身带有尊敬意义的尊敬词。
        \end{itemize}
        \begin{tabular}{lllll}
            \kr \ruby{例}{예}: &\kr 자다&$\to$&\kr 주무시다&睡觉\\
            &\kr 먹다 &$\to$&\kr 잡수시다&吃\\
            &\kr 있다 &$\to$&\kr 계시다&在
        \end{tabular}\\
        \begin{itemize}
            \item 表示尊敬时,还可以使用一些特定的尊敬词。
        \end{itemize}
        \begin{tabular}{lllll}
            \kr \ruby{例}{예}: &\kr 밥&$\to$& \kr 진지& 饭\\
            &\kr 말&$\to$&\kr 말씀& 话\\
            &\kr 집&$\to$& \kr \ruby{宅}{댁}& 府上
        \end{tabular}\\
        \begin{itemize}
            \item 后缀{\kr -님}用于表示人的称谓的体词后,表示尊敬。
        \end{itemize}
        \begin{tabular}{lllll}
            \kr \ruby{例}{예}:&\kr \ruby{先生}{선생}&$\to$&\kr  \ruby{先生}{선생}님 &老师/先生\footnote{其实还有医生的意思}\\
            &\kr \ruby{博士}{박사} &$\to$&\kr  \ruby{博士}{박사}님 &博士 \\
            &\kr \ruby{父母}{부모} &$\to$&\kr  \ruby{父母}{부모}님 &父母 \\
            &\kr 딸&$\to$&\kr  따님 &女儿 \\
            &\kr 아들 &$\to$&\kr  아드님 &儿子
        \end{tabular}\\
    \end{grammarsect}
\end{grammar}
\begin{grammar}
    \begin{grammarsect}[\kr -을/-를]
    \begin{itemize}
        \item 助词。表示宾语。体词词干以闭音节结尾时用{\kr -을},以开音节结尾时用{\kr -를}。
    \end{itemize}
    \begin{tabular}{lll}
        \kr \ruby{例}{예}: &\kr 그분은 \ruby{新聞}{신문}을 봅니다.&他在看报。\\
        &\kr 나는 밥을 잘 먹습니다.&我吃饭吃得很好。\\
        &\kr 이 아이는 \ruby{빵}{pão}을 좋아합니다.&这孩子喜欢面包。\\
        &\kr \ruby{學生}{학생}들이 노래를 잘 부릅니다.&学生们唱歌唱得很好。\\
        &\kr 나는 \ruby{親舊}{친구}를 기다립니다.&我在等朋友。
    \end{tabular}\\
    \end{grammarsect}
    \begin{grammarsect}[\kr -도]
        \begin{itemize}
            \item 补助词。表示包含、包括,类似于汉语的 “也"、“还” 等。有时也表示强调。
        \end{itemize}
        \begin{tabular}{lll}
            \kr \ruby{例}{예}&\kr 그분은 \ruby{韓國}{한국} 사람입니다. &他是韩国人。\\ 
            &\kr 저도 \ruby{韓國}{한국} 사람입니다.& 我也是韩国人。 \\
            &\kr \ruby{버스}{bus}가 많습니다.& 公共汽车很多。 \\
            &\kr \ruby{택시}{taxi}도 많습니다.& 出租车也很多。\\
            &\kr 나는 바지를 샀습니다. &我买了裤子。\\
            &\kr \ruby{구두}{くつ}도 샀습니다. &还买了皮鞋。\\
            &\kr 오늘은 \ruby{時間}{시간}이 없습니다. &今天没时间。\\
            &\kr 돈도 없습니다. &也没钱。 
        \end{tabular}\\
    \end{grammarsect}
    \begin{grammarsect}[\kr -고 싶다]
        \begin{itemize}
            \item 惯用型。用于动词词干后,表示愿望。用于主语是第一人称的陈述句和主语是第二人称的疑问句。类似于汉语的“想……”。
            {\color{gray} \item Ⅰ{\kr -고 싶다} }
        \end{itemize}
        \begin{tabular}{lll}
            \kr \ruby{例}{예}: &\kr 저는 \ruby{歷史}{역사}를 배우고 싶습니다.&我想学习历史。\\
            &\kr 나는 \ruby{人蔘茶}{인삼차}를 마시고 싶습니다.&我想喝人参茶。\\
            &\kr 가을에는 \ruby{旅行}{여행}을 가고 싶습니다.&秋天我想去旅游。\\
            &\kr \ruby{週末}{주말}에는 뭘 하고 싶습니까?&周末你想干什么?\\
            &\kr \ruby{故鄉}{고향}에 가고 싶습니까?&你想回老家吗?
        \end{tabular}\\
        \begin{itemize}
            \item 当主语是第三人称时用{\kr -고 싶어하다}或者在{\kr -고 싶다}后面加过去时制词尾{\kr 었/았}。
            {\color{gray} \item 过去式是Ⅲ-ㅆ-} 
        \end{itemize}
        \begin{tabular}{lll}
            \kr \ruby{例}{예}: &\kr 아이가 밖에 나가고 싶어합니다. &孩子想到外面去。\\
            &\kr 그 사람이 나를 만나고 싶어합니다. &那人想见我。\\
            &\kr 우리 \ruby{父母}{부모}님께서 서울에 오고 싶어하십니다.&我父母想来首尔。
        \end{tabular}\\
    \end{grammarsect}
\end{grammar}
\begin{grammar}
    \begin{grammarsect}[\kr -하고]
    \begin{itemize}
        \item 连接助词用于体词后面,连接两个体词。类似于汉语的“和”(参照\ref{gr})
    \end{itemize}
    \begin{tabular}{lll}
        \kr \ruby{例}{예}: &\kr 나는 \ruby{빵}{pão}하고 \ruby{牛乳}{우유}를 먹습니다.&我吃面包和牛奶。 \\
        &\kr 어머니는 \ruby{菓子}{과자}하고 과일을 사셨습니다. &妈妈买了点心和水果。\\ 
        &\kr \ruby{가방}{かばん}에 \ruby{冊}{책}하고 \ruby{空冊}{공책}이 있습니다。&包里有书和笔记本。\\
        &\kr \ruby{圖書館}{도서관}에 \ruby{新聞}{신문}하고 \ruby{雜誌}{잡지}가 많습니다.&图书馆里有很多报纸和杂志。\\
    \end{tabular}\\
    \end{grammarsect}
    \begin{grammarsect}[\kr -은/-는]
        \begin{itemize}
            \item 补助词。用于体词和部分助词、词尾后面,体词和助词、词尾的开音节后面用{\kr -는},闭音节后面用{\kr -은}。表示主题,即句子叙述的中心。用于体词后面时,常常表示句子的主语,或是强调的宾语。
        \end{itemize}
        \begin{tabular}{lll}
            \kr \ruby{例}{예}: &\kr 이분은 \ruby{金}{김} \ruby{先生}{선생}님입니다.&这位是金先生。\\
            &\kr 나는 김치는 싫습니다.&我不喜欢泡菜。\\
            &\kr 그 아이는 \ruby{運動}{운동}은 잘 합니다.&那孩子体育很好。\\
            &\kr 집에서는 \ruby{工夫}{공부}하지 않습니까?&在家不学习吗?\\
            &\kr \ruby{12}{열두}\ruby{時}{시}에는 \ruby{點心}{점심}을 먹습니다.&十二点吃午饭。
        \end{tabular}\\
        \begin{itemize}
            \item 有时也用于副词或词尾后面,表示部分否定。
        \end{itemize}
        \begin{tabular}{lll}
            \kr \ruby{例}{예}: &\kr \ruby{崔}{최} \ruby{先生}{선생}이 일을 잘은 합니다.&崔先生工作做得很好。\\
            &\kr \ruby{冊}{책}이 비싸지는 않습니다.&书不贵。
        \end{tabular}\\
    \end{grammarsect}
\end{grammar}
\section{\kr \ruby{類型}{유형} \ruby{練習}{연습}}
\begin{dic}
    \begin{dicsect}
        \begin{tabular}{rll}
            (보기)&\ruby{先生}{선생}:& \ruby{食堂}{식당}\\
            &\ruby{學生}{학생}:& \ruby{食堂}{식당}이 어디에 있습니까?\\
            1)&\ruby{先生}{선생}:&  \ruby{延世大學校}{연세대학교} \\
            &\ruby{學生}{학생}:& \ruby{延世大學校}{연세대학교}가 어디에 있습니까?\\
            2)&\ruby{先生}{선생}:& \ruby{市長}{시장} \\
            &\ruby{學生}{학생}:& \ruby{市長}{시장}이 어디에 있습니까?\\
            3)&\ruby{先生}{선생}:& \ruby{郵遞局}{우체국} \\
            &\ruby{學生}{학생}:& \ruby{郵遞局}{우체국}이 어디에 있습니까?\\
            4)&\ruby{先生}{선생}:& \ruby{銀行}{은행}\\ 
            &\ruby{學生}{학생}:& \ruby{銀行}{은행}이 어디에 있습니까?\\
            5)&\ruby{先生}{선생}:& \ruby{敎室}{교실}\\
            &\ruby{學生}{학생}:& \ruby{敎室}{교실}이 어디에 있습니까?\\
        \end{tabular}
    \end{dicsect}
\end{dic}