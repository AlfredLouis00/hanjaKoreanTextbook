\chapter{\kr \ruby{功夫}{공부}하기가 재미있습니다.}
\section{课文}
\subsection{对话}
{\kr \ruby{죤슨}{Johnson} 씨는 \ruby{敎室}{교실}에서 나와서 \ruby{金美善}{김미선} 씨를 만났다.\\}

{\kr
\begin{tabular}{lll}
    \ruby{죤슨}{Johnson} &:&\ruby{食堂}{식당}이 어디에 있습니까?\\
    \ruby{金美善}{김미선} &:& 저기에 있습니다.\\
    \ruby{죤슨}{Johnson} &:& \ruby{圖書館}{도서관}은 어디에 있습니까?\\
    \ruby{金美善}{김미선} &:&저 뒤에 있습니다. 같이 갑시다.\\
    \ruby{죤슨}{Johnson} &:& \ruby{未安}{미안}합니다.\\ 
    \ruby{金美善}{김미선} &:& 괜찮습니다.\\
\end{tabular}\\}

\noindent \textit{单词表}\\

\begin{tabular}{ll|ll|ll}
    \kr \ruby{食堂}{식당}&食堂,饭店&\kr 어디&什么地方,哪儿&\kr 저기&那儿(远称)\\
    \kr \ruby{圖書館}{도서관}&图书馆&\kr 저 뒤에&在那后面&\kr 같이&一起\\
    \kr 괜찮다&没关系
\end{tabular}\\
\subsection{对话}
{\kr 두 사람은 \ruby{圖書館}{도서관} 쪽으로간다.\\

\begin{tabular}{lll}
    \ruby{金美善}{김미선} &:& \ruby{功夫}{공부}하기가 어떻습니까?\\
    \ruby{죤슨}{Johnson} &:& 재미있습니다.\\
    \ruby{美善}{미선} &:& 누가 가르치십니까?\\
    \ruby{죤슨}{Johnson} &:& \ruby{朴}{박} \ruby{先生}{선생}님이 가르치십니다.\\
    \ruby{美善}{미선} &:& \ruby{學生}{학생}들이 \ruby{熱心}{열심}히 \ruby{功夫}{공부}합니까?\\
    \ruby{죤슨}{Johnson} &:& 예,\ruby{熱心}{열심}히 합니다.\\
\end{tabular}\\}

\noindent \textit{单词表} \\

\begin{tabular}{ll|ll|ll}
    \kr \ruby{功夫}{공부}하다 &学习 &\kr 어떻다& 怎么样 &\kr 누구 &谁\\
    \kr 가르치다 &教 &\kr\ruby{學生}{학생} &学生 &-들 &-们,-些\\
    \kr \ruby{熱心}{열심}히 &用功地,认真地&&&&(表示复数) 
\end{tabular}\\

\subsection{对话}
{\kr 두 사람이 가는데 어떤 \ruby{學生}{학생}이 \ruby{金美善}{김미선} 씨에게 \ruby{人事}{인사}를 하고 지나갔다.\\

\begin{tabular}{lll}
    \ruby{죤슨}{Johnson} &:& 그\ruby{分}{분}이 누구입니까?\\
    \ruby{美善}{미선}&:& \ruby{親舊}{친구}입니다.\\
    \ruby{죤슨}{Johnson} &:& 무엇을 \ruby{功夫}{공부}합니까?\\
    \ruby{美善}{미선}&:& \ruby{歷史}{역사}를 \ruby{功夫}{공부}합니다.\\
    \ruby{죤슨}{Johnson} &:& 저도 \ruby{歷史}{역사}를 배우고 싶습니다.\\
    \ruby{美善}{미선}&:& 그렇습니까?
\end{tabular}\\
}

\noindent \textit{单词表}\\

\begin{tabular}{ll|ll|ll}
    \kr 그\ruby{分}{분} &那位,他 &\kr \ruby{親舊}{친구} &朋友 &\ruby{歷史}{역사} &历史\\
    \kr 저 &我 (谦称)  &\kr -도 &……也 &\kr 배우다 &学习\\
    \kr -고 싶다. &想,希望 &\kr 그렇다 &那样,是那样
\end{tabular}\\

\subsection{对话}
{\kr 두 사람은 \ruby{圖書館}{도서관}으로 들어갔다.\\

\begin{tabular}{lll}
    \ruby{죤슨}{Johnson} &:& 이것은 무슨 \ruby{冊}{책}입니까?\\
    \ruby{美善}{미선} &:& \ruby{小說冊}{소설책}입니다.\\
    \ruby{죤슨}{Johnson} &:& \ruby{新聞}{신문}하고 \ruby{雜誌}{잡지}는 어디에 있습니까?\\
    \ruby{美善}{미선} &:& 저 \ruby{房}{방}에 있습니다.\\
    \ruby{죤슨}{Johnson} &:& 이제 나갑시다.\\
    \ruby{美善}{미선} &:& 좋습니다. 나갑시다.\\
\end{tabular}\\}

\noindent \textit{单词表}\\

\begin{tabular}{ll|ll|ll}
    \kr 무슨 &什么 &\kr\ruby{小說冊}{소설책} &小说(书) &\ruby{新聞}{신문} &报纸\\
    \kr \ruby{雜誌}{잡지} &杂志 &\ruby{房}{방} &屋子,房间 &이제 &现在\\
    \kr 나가다 &出去
\end{tabular}\\
\subsection{短文}
{\kr 나는 날마다 \ruby{圖書館}{도서관}에 갑니다.\\
\ruby{圖書館}{도서관}에는 \ruby{冊}{책}이 많습니다.\\
\ruby{學生}{학생}도 많습니다.\\
\ruby{學生}{학생}들이 \ruby{熱心}{열심}히 \ruby{功夫}{공부}합니다.\\
나는 \ruby{冊}{책}을 읽습니다.\\
\ruby{雜誌}{잡지}하고 \ruby{新聞}{신문}도 읽습니다.\\
\ruby{宿題}{숙제}도 합니다.\\}

\noindent \textit{单词表}\\

\begin{tabular}{ll|ll|ll}
    나 &我 &날마다 &每天 &읽다 &读
\end{tabular}\\
\section{\kr\ruby{文法}{문법}}
\begin{grammar}
    \begin{grammarsect}[\kr -에]
        \begin{itemize}
            \item 助词。用于表示地点的体词后,表示主体的存在处所,或趋向动词的目的地。
        \end{itemize}
        \begin{tabular}{lll}
            \kr\ruby{例}{예}:& \kr\ruby{英秀}{영수}는 집에 있습니다.&英秀在家。\\
            &\kr\ruby{父母}{부모}님이 \ruby{故鄉}{고향}에 계십니다.&父母在老家。\\
            &\kr\ruby{市場}{시장}에 \ruby{物件}{물건}이 많습니다.&市场里有很多东西。\\
            &\kr 날마다 \ruby{圖書館}{도서관}에 갑니다.&每天去图书馆。\\
            &\kr 어디에 가십니까?&您去哪儿?\\
            &\kr\ruby{來日}{내일} 우리 집에 오십시오.&明天来我家吧。
        \end{tabular}\\
        \begin{itemize}
            \item 表示存在处所的助词{\kr-에}加上动词{\kr 있다},构成{\kr -에 있다}的句型,表示$\times \times$在$\times \times$ (某处)。
        \end{itemize}
        \begin{tabular}{lll}
            \kr \ruby{例}{예}:&\kr \ruby{冊}{책}이 \ruby{冊床}{책상}에 있습니다.&书在桌子上。\\ 
            &\kr \ruby{圖書館}{도서관}이 어디에 있습니까?&图书馆在哪儿?\\
            &\kr \ruby{同生}{동생}이 \ruby{美國}{미국}에 있습니다.&弟弟在美国。\\
            &\kr \ruby{學生}{학생}들이 \ruby{教室}{교실}에 없습니다.&学生们不在教室\\
            &\kr 어머니는 집에 계십니다.&妈妈在家。\\
        \end{tabular}\\
    \end{grammarsect}
\end{grammar}