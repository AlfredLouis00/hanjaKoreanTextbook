\chapter{\kr 뭘 드시겠습니까?}
\section{课文}
\subsection{对话}
{\kr \ruby{金美善}{김미선} 씨와 \ruby{존슨}{Johnson} 씨는 \ruby{圖書館}{도서관}에서 나왔다.\\}
{\kr \begin{tabular}{lll}
		\ruby{金美善}{김미선}    & : & \ruby{只今}{지금} 몇 \ruby{時}{시}입니까? \\
		\ruby{죤슨}{Johnson} & : & 한 \ruby{時}{시}입니다.               \\
		\ruby{金美善}{김미선}    & : & 배가 고픕니다.                        \\
		\ruby{죤슨}{Johnson} & : & \ruby{食堂}{식당}에 갑시다.             \\
		\ruby{金美善}{김미선}    & : & 무슨 \ruby{飲食}{음식}을 좋아하십니까?       \\
		\ruby{죤슨}{Johnson} & : & \ruby{韓食}{한식}을 좋아합니다.           \\
	\end{tabular}\\}
\textit{单词表}\\
\begin{tabular}{llllll}
	\ruby{只今}{지금} & 现在 & 몇             & 几, 多少 & \ruby{時}{시}   & (时间的)点,时 \\
	배             & 肚子 & 고프다           & 饿     & \ruby{飲食}{음식} & 饭、食物、饮食  \\
	좋아하다          & 喜欢 & \ruby{韓食}{한식} & 韩餐
\end{tabular}\\
\subsection{对话}
{\kr 두 사람은 \ruby{韓食}{한식}집에 들어갔다.\\}
{\kr \begin{tabular}{lll}
		\ruby{金美善}{김미선}    & : & 덥지요?                           \\
		\ruby{죤슨}{Johnson} & : & 예, 덥습니다.                       \\
		\ruby{金美善}{김미선}    & : & 여보세요 여기 물 좀 주십시오.              \\
		\ruby{죤슨}{Johnson} & : & 뭘 잡수시겠습니까?                     \\
		\ruby{金美善}{김미선}    & : & 저는 \ruby{冷麵}{냉면}을 먹겠습니다.       \\
		\ruby{죤슨}{Johnson} & : & 그럼, 불고기하고 \ruby{冷麵}{냉면}을 시킵시다. \\
	\end{tabular}\\}
\textit{单词表} \\
\begin{tabular}{llllll}
	덥다   & 热      & 물             & 水  & 주다  & 给          \\
	잡수시다 & 吃 (敬语) & \ruby{冷麵}{냉면} & 冷面 & 먹다  & 吃          \\
	그럼   & 那么     & 불고기           & 烤肉 & 시키다 & 要(菜), 点(菜)
\end{tabular}\\
\subsection{对话}
{\kr \begin{tabular}{lll}
		\ruby{金美善}{김미선}    & : & 불고기 맛이 어떻습니까? \\
		\ruby{죤슨}{Johnson} & : & 참 맛이 있습니다.    \\
		\ruby{金美善}{김미선}    & : & 이것 좀 잡수십시오.   \\
		\ruby{죤슨}{Johnson} & : & 그것이 무엇입니까?    \\
		\ruby{金美善}{김미선}    & : & 오이김치입니다.      \\
		\ruby{죤슨}{Johnson} & : & 맵습니까?         \\
		\ruby{金美善}{김미선}    & : & 아니요, 맵지 않습니다. \\
	\end{tabular}\\}
\textit{单词表} \\
\begin{tabular}{llllllll}
	맛 & 味道 & 참 & 真 & 오이 & 黄瓜 & 맵다 & 辣 \\
\end{tabular}\\
\subsection{对话}
{\kr \begin{tabular}{lll}
		\ruby{金美善}{김미선}    & : & 불고기를 더 시킬까요?                                     \\
		\ruby{죤슨}{Johnson} & : & 아니요,많이 먹었습니다.                                    \\
		\ruby{金美善}{김미선}    & : & 그럼,\ruby{茶}{차}를 마십시다.                            \\
		\ruby{죤슨}{Johnson} & : & \ruby{人蔘茶}{인삼차}가 어떻습니까?                          \\
		\ruby{金美善}{김미선}    & : & 저는 \ruby{커피}{coffee}를 마시겠습니다.                    \\
		\ruby{죤슨}{Johnson} & : & 여보세요, \ruby{人蔘茶}{인삼차}하고 \ruby{커피}{coffee}를 주십시오. \\
	\end{tabular}\\}
\textit{单词表} \\
\begin{tabular}{llllllll}
	더 & 再 & \ruby{茶}{차} & 茶 & 마시다 & 喝 & \ruby{人蔘茶}{인삼차} & 人参茶 \\
\end{tabular}\\
\subsection{课文}
나는 한 \ruby{時}{시}에 \ruby{點心}{점심}을 먹습니다.\\\indent
나는 \ruby{韓國}{한국} \ruby{飲食}{음식}을 좋아합니다.\\\indent
오늘은 \ruby{親舊}{친구}와 같이  \ruby{食堂}{식당}에 갔습니다.\\\indent
비빔밥하고 \ruby{冷麵}{냉면}을 시켰습니다.\\\indent
그 집  \ruby{飲食}{음식}이 참 맛이 있었습니다.\\\indent
값도 비싸지 않았습니다.\\\indent
아가씨도 \ruby{親切}{친절}했습니다.\\
\textit{单词表} \\
\begin{tabular}{llllll}
	\ruby{點心}{점심} & 午饭 & 오늘  & 今天 & 값               & 价格,价钱 \\
	비싸다           & 贵  & 아가씨 & 小姐 & \ruby{親切}{친절}하다 & 热情,亲切
\end{tabular}\\
\section{\kr \ruby{文法}{문법}}
\begin{grammar}
	\begin{grammarsect}[韩国语的固有数词\\]
		\begin{tabular}{rlrlrlrlrlrlrlrlrlrlrl}
			1  & 하나 & 2  & 둘   & 3  & 셋  & 4  & 넷 & 5  & 다섯 & 6  & 여섯 & 7  & 일곱 & 8  & 여넓 & 9   & 아홉          \\
			10 & 열  & 11 & 열하나 & 12 & 열둘                                                                      \\
			20 & 스물 & 30 & 서른  & 40 & 마혼 & 50 & 쉰 & 60 & 예순 & 70 & 일혼 & 80 & 여든 & 90 & 아혼 & 100 & \ruby{百}{백}
		\end{tabular}\\
		\begin{itemize}
			\item 韩国语的部分固有数词与量词(包括傲量词用的名词)连用时要产生音变现象, 这时하나、둘、셋、넷要变成한、두、세、네。
		\end{itemize}
		\begin{tabular}{lllllll}
			\ruby{例}{예}: & 한 \ruby{時}{시}  & 두 \ruby{時}{시}  & 세 \ruby{時}{시}  & 네 \ruby{時}{시} & 다섯 \ruby{時}{시} & 여섯 \ruby{時}{시} \\
			             & 일곱 \ruby{時}{시} & 여덟 \ruby{時}{시} & 아홉 \ruby{時}{시} & 열 \ruby{時}{시} & 열한 \ruby{時}{시} & 열두 \ruby{時}{시} \\
		\end{tabular}\\
	\end{grammarsect}
\end{grammar}
\begin{grammar}
	\begin{grammarsect}[-겠-]
		\begin{itemize}
			\item 时制词尾。表示未来时, 用于谓词词干和体词谓词形后,根据主语人称的不同,可以表示意志、推测、可能等。
			\item {\color{gray} Ⅰ-겠-}
		\end{itemize}
		\begin{tabular}{lll}
			\ruby{例}{예}: & 오늘 저는 집에 있겠습니다.             & 今天我要待在家里。 \\
			             & \ruby{來日}{내일} 다시 오겠습니다.     & 我明天再来。    \\
			             & 저녁에는 \ruby{親舊}{친구}를 만나겠습니다. & 我晚上要见朋友。  \\
			             & 오늘 밤에 \ruby{電話}{전화}하겠습니다.   & 我今天夜里打电话。
		\end{tabular}\\
	\end{grammarsect}
	\begin{grammarsect}[-지요]
		\begin{itemize}
			\item 准敬阶终结词尾。用于谓词词干和体词谓词形后。当用于疑问句时,表示说话人对所提问题已有所知,只是希望得到对方的确认,类似于汉语的“……吧?”。
			\item 在口语中,由句尾语调的升降,决定句子表示陈述、疑问、命令、共动等句式。
			\item {\color{gray} Ⅰ-지-요。Ⅰ-지形成不定阶终结词尾,加入요成为准敬阶终结词尾。有时지요缩写成죠。}
		\end{itemize}
		\begin{tabular}{lll}
			\ruby{例}{예}: & 그 일은 제가 하지요.          & 那事我做吧。  \\
			             & 제가 돈을 내지요.            & 我来付钱吧。  \\
			             & 날씨가 덥지요?              & 天气热吧?   \\
			             & \ruby{先生}{선생}님이 좋지요?  & 老师很好吧?  \\
			             & 먼저 가시지요.              & 请先走吧。   \\
			             & 더 잡수시지요.              & 请再吃点儿吧。 \\
			             & 같이 가시지요.              & 一起去吧。   \\
			             & \ruby{來日}{내일} 또 만나지요. & 明天再见吧。  \\
		\end{tabular}\\
	\end{grammarsect}
\end{grammar}
\begin{grammar}
	\begin{grammarsect}[-지 않다]
		\begin{itemize}
			\item 惯用型。用于谓词词干后,表示否定。不能用于命令句和共动句。
			\item {\color{gray} Ⅰ-지 않다.}
		\end{itemize}
		\begin{tabular}{lll}
			\ruby{例}{예}: & \ruby{來日}{내일}은 \ruby{學校}{학교}에 가지 않습니다.   & 明天不去学校。  \\
			             & 그 \ruby{學生}{학생}은 \ruby{宿題}{숙제}를 하지 않습니다. & 那学生不傲作业。 \\
			             & 요즘은 바쁘지 않습니까?                            & 最近不忙吗?   \\
			             & \ruby{氣分}{기분}이 좋지 않습니다.                  & 情绪不好。    \\
		\end{tabular}\\
	\end{grammarsect}
	\begin{grammarsect}[안]
		\begin{enumerate}
			\item 副词。表示否定,主要用在动词前面。
		\end{enumerate}
		\begin{tabular}{lll}
			\ruby{例}{예}: & 그것은 안 사겠습니다.                & 不买那东西。  \\
			             & \ruby{牛乳}{우유}를 안 마십니다.      & 不喝牛奶。   \\
			             & \ruby{라디오}{radio}를 안 들으십니까? & 不听收音机吗? \\
		\end{tabular}\\
	\end{grammarsect}
	\begin{enumerate}
		\item 韩国语中有很多由名词하다构成的动词,可以在名词和하다之间加上안,构成对这类动词的否定。在这种情况下,一般要在名词后面加上表示宾语的助词-올/룰。
	\end{enumerate}
	\begin{tabular}{lll}
		\ruby{例}{예}예: & 그 사람은 말을 안 합니다.             & 那个人不说话。 \\
		              & 아이들이 \ruby{工夫}{공부}를 안 합니다.  & 孩子们不学习。 \\
		              & 요즘은 \ruby{運動}{운동}운동을 안 합니다. & 最近没有锻炼。 \\
	\end{tabular}\\
\end{grammar}
\begin{grammar}
	\begin{grammarsect}[ -(으)ㄹ까요]
		\begin{itemize}
			\item 准敬阶疑问式终结词尾。表示意图和推测。当主语是第一人称单数时,表示说话人征求听话人的意见。当主 语是第一人称复数时,表示建议听话人一起行动,语气比共动句婉转。
			\item 谓词词干末音节是开音节时用-ㄹ까요, 闭音节时用-올까요。
			\item {\color{gray} Ⅱ*-ㄹ까-요. -요是不定阶转准敬阶的终结词尾,-ㄹ까是不定阶的终结词尾,两者叠加就是准敬阶疑问式终结词尾}
		\end{itemize}
		\begin{tabular}{lll}
			\ruby{例}{예}: & (제가) \ruby{電話}{전화}를 할까요?     & (我) 打个电话好吗?   \\
			             & (제가) \ruby{門}{문}을 닫을까요?      & (我) 把门关上好吗?   \\
			             & (우리가) 무엇을 시킬까요?              & (我们)点些什么呢?    \\
			             & (우리가) \ruby{來日}{내일} 일찍 만날까요? & (我们)明天早ᅳ点见面吧? \\
		\end{tabular}\\
		\begin{enumerate}
			\item 当主语是第一人称单数时,回答时用-으십시오/십시오,当主语是第一人称复数时,回答时用-읍시다/ㅂ시다。
		\end{enumerate}
	\end{grammarsect}
	\begin{grammarsect}[时制]
		\begin{enumerate}
			\item 和其他语言ᅳ样, 韩国语的时制也是根据自然时间, 分为现在时、过去时和将来时。此外,韩国语的时态还可以表现说话者的意图和态度。谓词词干和体词谓词形后面加上-았-/-었-/-였-/-겠-{\color{gray} (语基的说法是Ⅲ-ㅆ-和Ⅰ-겠-分别表示过去时和将来时)}表示时制。
		\end{enumerate}{\color{white} 这些是占位用的}
		\begin{tabular}{|l|c|c|c|}
			\hline
			\diagbox{用法{\color{white} 啊啊啊啊啊啊}}{时制} & 过去时 & 现在时                   & 将来时                  \\\hline
			谓词词干末音节是-아, -오时                        & -았- & \multirow{3}{*}{-\ -} & \multirow{3}{*}{-겠-} \\\cline{1-2}
			谓词词干末音节是-어, -우, -요, -이                 & -었- &                       &                      \\\cline{1-2}
			谓词词干末音节是하-时                            & -였- &                       &                      \\\hline
		\end{tabular}\\
		\begin{tabular}{lll}
			\ruby{例}{예}: & \ruby{朴}{박} \ruby{先生}{선생}님은 요즘 바쁘십니다. & 朴先生最近很忙。  \\
			             & \ruby{學生}{학생}들이 \ruby{冊}{책}책을 읽습니다.   & 学生们在读书。   \\
			             & 나는 어제 \ruby{親舊}{친구}를 만났습니다.           & 我昨天见了朋友。  \\
			             & 그 \ruby{映畫}{영화}는 재미있었습니다.             & 那个电影很有意思。 \\
			             & \ruby{週末}{주말}에 \ruby{旅行}{여행}을 떠나겠습니다. & 周末要去旅行。   \\
			             & \ruby{來日}{내일}은 비가 오겠습니다.              & 明天要下雨。
		\end{tabular}\\
	\end{grammarsect}
	\begin{grammarsect}[-았-/-었-/-였-]
		\begin{enumerate}
			\item 过去时时制词尾。谓词词干末音节是-아、-오时用-았-,谓词词干末音节是其他元音时用-었-,谓词词干末音节是하-时用-였-。谓词词干和时制词尾-았-/-었-/-였-结合时,发音和拼写要发生如下变化:
		\end{enumerate}
		{\color{white} 過去點地方}
		\begin{tabular}{|ccccc|ccccccccc|}
			\hline
			ㅏ & + & 았 & $\to$ & 았 & 가   & + & 았 & + & 다 & $\to$ & 가았다   & $\to$ & 갔다   \\
			  &   &   &       &   & 사   & + & 았 & + & 다 & $\to$ & 사았다   & $\to$ & 샀다   \\
			ㅓ & + & 었 & $\to$ & 었 & 서   & + & 었 & + & 다 & $\to$ & 서었다   & $\to$ & 섰다   \\
			ㅡ & + & 었 & $\to$ & 었 & 쓰   & + & 었 & + & 다 & $\to$ & 쓰었다   & $\to$ & 썼다   \\\hline
			ㅗ & + & 았 & $\to$ & 왔 & 오   & + & 았 & + & 다 & $\to$ & 오았다   & $\to$ & 왔다   \\
			  &   &   &       &   & 보   & + & 았 & + & 다 & $\to$ & 보았다   & $\to$ & 봤다   \\
			ㅜ & + & 었 & $\to$ & 웠 & 배우  & + & 었 & + & 다 & $\to$ & 배우었다  & $\to$ & 배웠다  \\
			ㅣ & + & 었 & $\to$ & 였 & 가르치 & + & 었 & + & 다 & $\to$ & 가르치었다 & $\to$ & 가르쳤다 \\\hline
		\end{tabular}\\

		\begin{tabular}{lll}
			\ruby{例}{예}: & 밥을 많이 먹었습니다.                                        & 吃了很多饭。      \\
			             & 어제 집에 있었습니다.                                        & 昨天在家了。      \\
			             & \ruby{金}{김} \ruby{先生}{선생}님은 집에 \ruby{電話}{전화}를 했습니다. & 金先生给家里打了电话。 \\
			             & 어제 밤에 비가 왔습니다.                                      & 昨天夜里下雨了。    \\
			             & 그 사람은 어제 \ruby{美國}{미국}에 갔습니다.                       & 那人昨天去美国了。   \\
			             & 어머니께 \ruby{便紙}{편지}를 썼습니다.                           & 给母亲写了信。     \\
		\end{tabular}\\
	\end{grammarsect}
\end{grammar}
\begin{grammar}
	\begin{grammarsect}[-에]
		\begin{itemize}
			\item 助词。用于表示时间的体词后,表示时间(参照2.5 G1\footnote{这个参考不存在,原书就没有2.5 G1})。\\
		\end{itemize}
		\begin{tabular}{lll}
			\ruby{例}{예}: & 몇 시에 \ruby{點心}{점심}을 먹습니까?              & 几点吃午饭?     \\
			             & 아침에 만납시다.                              & 早上见吧。      \\
			             & \ruby{週末}{주말}에 \ruby{親舊}{친구} 집에 가겠습니다. & 周末打算去朋友家。  \\
			             & 여름에는 바다에서 지냈습니다.                       & 夏天是在海滨度过的。 \\
			             & 밤 열한 \ruby{時}{시}에 잤습니다.                & 晚上 11 点睡的。
		\end{tabular}\\
	\end{grammarsect}
	\begin{grammarsect}[-과 / -와]\label{grm:ua}
		\begin{enumerate}
			\item 连接助词。用于体词和体词的连接,表示并列关系。类似于汉语的“和、与、同、跟”等。体词词干末音节是开音节时用-와, 闭音节时用-과。在口语中,可以用-하고代替-과/-와 (参照\ref{grm:hako})。
		\end{enumerate}
		{\large \HandRight -과 같이 / -와 같이\\}
		\begin{enumerate}
			\item 惯用型。用于体词后边,表示一起进行某一行为。\\
		\end{enumerate}
		\begin{tabular}{lll}
			\ruby{例}{예}: &\ruby{金}{김} \ruby{先生}{선생}과 같이 이야기를 했습니다. &和金先生一起谈话了。\\
			&그분은 \ruby{夫人}{부인}과 같이 \ruby{濟州島}{제주도}에 갔습니다. &他和夫人一起去济州岛了。 \\
			&누구와 같이 \ruby{點心}{점심}을 먹습니까?&和谁一起吃午饭呢? \\
			&우리는 어머니와 같이 삽니다.&我们和妈妈一起生活。\\
			&어제 \ruby{親舊}{친구}와 같이 \ruby{映畫}{영화}를 보았습니다.&昨天和朋友一起看了电影。\\
	\end{tabular}\\
	\end{grammarsect}
\end{grammar}
\section{\kr \ruby{類型}{유형} \ruby{練習}{연습}}
{\kr 
\begin{dic}
	\begin{dicsect}
		\begin{tabular}{rll}
			(보기) &\ruby{先生}{선생}:& \ruby{只今}{지금} 몇 \ruby{時}{시}입니까? (한 \ruby{時}{시}) \\
			&\ruby{學生}{학생}:& \ruby{只今}{지금} 한 \ruby{時}{시}입니다.\\
			\con&\ruby{先生}{선생}:& \ruby{只今}{지금} 몇 \ruby{時}{시}입니까? (세 \ruby{時}{시} \ruby{半}{반}) \\
			&\ruby{學生}{학생}:& \ruby{只今}{지금} 세 \ruby{時}{시} \ruby{半}{반}입니다.\\
			\con&\ruby{先生}{선생}:& 날마다 몇 \ruby{時間}{시간} 공부합니까? (네 \ruby{時間}{시간}) \\
			&\ruby{學生}{학생}:& 날마다 네 \ruby{時間}{시간} 공부합니다.\\
			\con&\ruby{先生}{선생}:& 몇 사람입니까? (두 사람) \\
			&\ruby{學生}{학생}:& 두 사람입니다.\\
			\con&\ruby{先生}{선생}:& 몇 \ruby{個}{개}입니까? (여덟 \ruby{個}{개}) \\
			&\ruby{學生}{학생}:& 여덟 \ruby{個}{개}입니다.\\
			\con&\ruby{先生}{선생}:& 나이가 몇 살입니까? (스무 살) \\
			&\ruby{學生}{학생}:& 스무 살입니다.\\
		\end{tabular}\\
	\end{dicsect}
	\begin{dicsect}
		\begin{tabular}{rll}
			(보기) &\ruby{先生}{선생}:& 배가 고품니다. (식당에 갑니다) \\
			&\ruby{學生}{학생}:& 아! 그렇습니까? 식당에 갑시다.\\
			\con&\ruby{先生}{선생}:& 여행하고싶습니다. (제주도에 갑니다) \\
			&\ruby{學生}{학생}:& 아! 그렇습니까?제주도에 갑시다.\\
			\con&\ruby{先生}{선생}:& 피곤합니다. (쉽니다) \\
			&\ruby{學生}{학생}:& 아! 그렇습니까? 쉽시다.\\
			\con&\ruby{先生}{선생}:& 배가 아품니다. (병원에 갑니다) \\
			&\ruby{學生}{학생}:& 아! 그렇습니까? 병원에 갑시다.\\
			\con&\ruby{先生}{선생}:& 한국말이 어렵습니다. (열심히 연습합니다) \\
			&\ruby{學生}{학생}:& 아! 그렇습니까? 열심히 연습합시다.\\
			\con&\ruby{先生}{선생}:& 김 선생님을보고싶습니다. (만납니다) \\
			&\ruby{學生}{학생}:& 아! 그렇습니까?만납시다.\\
		\end{tabular}\\
	\end{dicsect}
	\begin{dicsect}
		\begin{tabular}{rll}
			(보기) &\ruby{先生}{선생}:& 음식 / 좋아합니다.\\
			&\ruby{學生}{학생}1:& 무슨 음식을 좋아하십니까? (한식) \\
			&\ruby{學生}{학생}2:& 한식을 좋아합니다.\\
			\con&\ruby{先生}{선생}:& 운동 / 좋아합니다.\\
			&\ruby{學生}{학생}1:& 무슨 운동을 좋아하십니까? (야구) \\
			&\ruby{學生}{학생}2:& 야구를 좋아합니다.\\
			\con&\ruby{先生}{선생}:& 색 / 좋아합니다.\\
			&\ruby{學生}{학생}1:& 무슨 색을 좋아하십니까? (빨간색) \\
			&\ruby{學生}{학생}2:& 빨간색을 좋아합니다.\\
			\con&\ruby{先生}{선생}:& 꽃 / 좋아합니다.\\
			&\ruby{學生}{학생}1:& 무슨 꽃을 좋아하십니까? (장미꽃) \\
			&\ruby{學生}{학생}2:& 장미꽃을 좋아합니다.\\
			\con&\ruby{先生}{선생}:& 옷 / 좋아합니다.\\
			&\ruby{學生}{학생}1:& 무슨 옷을 좋아하십니까? (청바지) \\
			&\ruby{學生}{학생}2:& 청바지를 좋아합니다.\\
			\con&\ruby{先生}{선생}:& 차 / 좋아합니다.\\
			&\ruby{學生}{학생}1:& 무슨 차를 좋아하십니까? (인삼차) \\
			&\ruby{學生}{학생}2:& 인삼차를 좋아합니다.\\
		\end{tabular}\\
	\end{dicsect}
\end{dic}
\begin{dic}
	\begin{dicsect}
		\begin{tabular}{rll}
			(보기) &\ruby{先生}{선생}:& 덥습니다.\\
		&\ruby{學生}{학생}:& 덥지요?\\
		\con&\ruby{先生}{선생}:& 한국말이 어렵습니다.\\
		&\ruby{學生}{학생}:& 한국말이 어렵지요?\\
		\con&\ruby{先生}{선생}:& 불고기를 좋아하십니다.\\
		&\ruby{學生}{학생}:& 불고기를 좋아하시지요?\\
		\con&\ruby{先生}{선생}:& 요즘 바쁘십니다.\\
		&\ruby{學生}{학생}:& 요즘 바쁘시지요?\\
		\con&\ruby{先生}{선생}:& 저 아이가 예뽑니다.\\
		&\ruby{學生}{학생}:& 저 아이가 예쁘지요?\\
		\con& \ruby{先生}{선생}:& 김치를 잘 잡수십니다.\\
		&\ruby{學生}{학생}:& 김치를 잘 잡수시지요?\\
		\end{tabular}\\
	\end{dicsect}
	\begin{dicsect}
		\begin{tabular}{rll}
			(보기) &\ruby{先生}{선생}:& 덥지요? (예) \\
&\ruby{學生}{학생}:& 예, 덥습니다.\\
\con&\ruby{先生}{선생}:& 한국말이 어렵지요? (예) \\
&\ruby{學生}{학생}:& 예, 한국말이 어렵습니다.\\
\con&\ruby{先生}{선생}:& 불고기를 좋아하시지요? (예) \\
&\ruby{學生}{학생}:& 예,불고기를 좋이"합니다.\\
\con&\ruby{先生}{선생}:& 요즘 바쁘시지요? (예) \\
&\ruby{學生}{학생}:& 예, 요즘 바뽑니다.\\
\con&\ruby{先生}{선생}:& 저 아이가 예쁘지요? (예) \\
&\ruby{學生}{학생}:& 예, 저 아이가 예뽑니다.\\
\con&\ruby{先生}{선생}:& 김치를 잘 잡수시지요? (예) \\
&\ruby{學生}{학생}:& 예, 김치를 잘 먹습니다.\\
		\end{tabular}\\
	\end{dicsect}
	\begin{dicsect}
		\begin{tabular}{rll}
			(보기) &\ruby{先生}{선생}:& 물\\
&\ruby{學生}{학생}:& 여보세요 여기 물 좀 주십시오.\\
\con&\ruby{先生}{선생}:& 커피 \\
&\ruby{學生}{학생}:& 여보세요 여기 커피 좀 주십시오.\\
\con&\ruby{先生}{선생}:& 아이스크림하고 빵\\
&\ruby{學生}{학생}:& 여보세요 여기 아이스크림하고 빵 좀 주십시오.\\
\con&\ruby{先生}{선생}:& 불고기하고 냉면\\
&\ruby{學生}{학생}:& 여보세요 여기 불고기하고 냉면 좀 주십시오.\\
\con&\ruby{先生}{선생}:& 맥주하고 안주\\
&\ruby{學生}{학생}:& 여보세요 여기 맥주하고 안주 좀 주십시오.\\
\con&\ruby{先生}{선생}:& 종이하고 볼펜\\
&\ruby{學生}{학생}:& 여보세요 여기 종이하고 볼펜 좀 주십시오.\\
		\end{tabular}\\
	\end{dicsect}
	\begin{dicsect}
		\begin{tabular}{rll}
			(보기) &\ruby{先生}{선생}:& 뭘 잡수십니까?\\
&\ruby{學生}{학생}:& 뭘 잡수시겠습니까?\\
\con&\ruby{先生}{선생}:& 뭘 하십니까?\\
&\ruby{學生}{학생}:& 뭘 하시겠습니까?\\
\con&\ruby{先生}{선생}:& 어디에 가십니까?\\
&\ruby{學生}{학생}:& 어디에 가시겠습니까?\\
\con&\ruby{先生}{선생}:& 누가 읽으십니까?\\
&\ruby{學生}{학생}:& 누가 읽으시겠습니까?\\
\con&\ruby{先生}{선생}:& 누구를 만나십니까?\\
&\ruby{學生}{학생}:& 누구를 만나시겠습니까?\\
\con&\ruby{先生}{선생}:& 그분이 노래를 하십니까?\\
&\ruby{學生}{학생}:& 그분이 노래를 하시겠습니까?\\
		\end{tabular}\\
	\end{dicsect}
\end{dic}
\begin{dicsect}
	\begin{tabular}{rll}
		(보기) &\ruby{先生}{선생}:& 뭘 잡수시겠습니까? (냉면) \\
&\ruby{學生}{학생}:& 냉면을 먹겠습니다.\\
\con&\ruby{先生}{선생}:& 어디에 가시겠습니까? (시내) \\
&\ruby{學生}{학생}:& 시내에 가겠습니다.\\
\con&\ruby{先生}{선생}:& 누구를만나시겠습니까? (고향친구) \\
&\ruby{學生}{학생}:& 고향 친구를 만나겠습니다.\\
\con&\ruby{先生}{선생}:& 몇 \ruby{時間}{시간}주무시겠습니까? (일곱\ruby{時間}{시간}) \\
&\ruby{學生}{학생}:& 일곱 \ruby{時間}{시간} 자겠습니다.\\
\con&\ruby{先生}{선생}:& 무슨 책을 읽으시겠습니까? (소설책) \\
&\ruby{學生}{학생}:& 소설책을 읽겠습니다.\\
\con&\ruby{先生}{선생}:& 그 분 이 노래를 하시겠습니까? (예) \\
&\ruby{學生}{학생}:& 예, 그 분이 노래를 하시겠습니다.\\
	\end{tabular}\\
\end{dicsect}
\begin{dic}
	\begin{dicsect}
		\begin{tabular}{rll}
			(보기) &\ruby{先生}{선생}:& 불고기 맛이 어떻습니까? (맛이 있습니다) \\
&\ruby{學生}{학생}:& 참 맛이 있습니다.\\
\con&\ruby{先生}{선생}:& 김치 맛이 어떻습니까? (맵습니다) \\
&\ruby{學生}{학생}:& 참 맵습니다. \\
\con&\ruby{先生}{선생}:& 이 사과 맛이 어떻습니까? (십니다) \\
&\ruby{學生}{학생}:& 참 십니다. \\
\con&\ruby{先生}{선생}:& 이 과자 맛이 어떻습니까? (답니다) \\
&\ruby{學生}{학생}:& 참 답니다. \\
\con&\ruby{先生}{선생}:& 약 맛이 어떻습니까? (씁니다) \\
&\ruby{學生}{학생}:& 참 씁니다. \\
\con&\ruby{先生}{선생}:& 이 반찬 맛이 어떻습니까? (짭니다)\\
&\ruby{學生}{학생}:& 참 짭니다. \\
		\end{tabular}\\
	\end{dicsect}
	\begin{dicsect}
		\begin{tabular}{rll}
			(보기) &\ruby{先生}{선생}:& 김치가 맵습니까? (아니요) \\
&\ruby{學生}{학생}:& 아니요,맵지 않습니다.\\
\con&\ruby{先生}{선생}:& 교실이 덥습니까? (아니요) \\
&\ruby{學生}{학생}:& 아니요,덥지 않습니다.\\
\con&\ruby{先生}{선생}:& 내일도 학교에 갑니까? (아니요) \\
&\ruby{學生}{학생}:& 아니요,가지 않습니다.\\
\con&\ruby{先生}{선생}:& 요즘도 바쁘십니까? (아니요) \\
&\ruby{學生}{학생}:& 아니요, 바쁘지 않습니다.\\
\con&\ruby{先生}{선생}:& 담배를 피우십니까? (아니요) \\
&\ruby{學生}{학생}:& 아니요,피우지 않습니다.\\
\con&\ruby{先生}{선생}:& 냉면을 잡수시겠습니까? (아니요) \\
&\ruby{學生}{학생}:& 아니요,먹지 않겠습니다.\\
		\end{tabular}\\
	\end{dicsect}
	\begin{dicsect}
		\begin{tabular}{rll}
			(보기) &\ruby{學生}{학생}:& 그것은 사지 않겠습니다.\\
			&\ruby{先生}{선생}:& 그것은 안 사겠습니다.\\
\con&\ruby{先生}{선생}:& 뉴스를 듣지 않습니다.\\
&\ruby{學生}{학생}:& 뉴스를 안 듣습니다.\\
\con&\ruby{先生}{선생}:& 그 사람은 말을 하지 않습니다.\\
&\ruby{學生}{학생}:& 그 사람은 말을 안 합니다.\\
\con&\ruby{先生}{선생}:& 저는 마늘을 먹지 않습니다.\\
&\ruby{學生}{학생}:& 저는 마늘을 안 먹습니다.\\
\con&\ruby{先生}{선생}:& 학교에 가지 않습니다.\\
&\ruby{學生}{학생}:& 학교에 안 갑니다.\\
\con&\ruby{先生}{선생}:& 저는 여자 친구를 만나지 않겠습니다.\\
&\ruby{學生}{학생}:& 저는 여자 친구를 안 만나겠습니다.\\
		\end{tabular}\\
	\end{dicsect}
\end{dic}
}