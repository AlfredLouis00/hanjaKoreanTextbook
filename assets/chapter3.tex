\chapter{\kr 뭘 드시겠습니까?}
\section{课文}
\subsection{对话}
{\kr \ruby{金美善}{김미선} 씨와 \ruby{존슨}{Johnson} 씨는 \ruby{圖書館}{도서관}에서 나왔다.\\}
{\kr \begin{tabular}{lll}
    \ruby{金美善}{김미선}&:& \ruby{只今}{지금} 몇 \ruby{時}{시}입니까?\\
    \ruby{죤슨}{Johnson}&:& 한 \ruby{時}{시}입니다.\\
    \ruby{金美善}{김미선}&:& 배가 고픕니다.\\
    \ruby{죤슨}{Johnson}&:& \ruby{食堂}{식당}에 갑시다.\\
    \ruby{金美善}{김미선}&:& 무슨 \ruby{飲食}{음식}을 좋아하십니까?\\
    \ruby{죤슨}{Johnson}&:& \ruby{韓食}{한식}을 좋아합니다.\\
\end{tabular}\\}
\textit{单词表}\\
\begin{tabular}{llllll}
    \ruby{只今}{지금}&现在&몇&几, 多少&\ruby{時}{시}&(时间的)点,时\\
    배&肚子&고프다&饿&\ruby{飲食}{음식}&饭、食物、饮食\\
    좋아하다&喜欢&\ruby{韓食}{한식}&韩餐
\end{tabular}\\
\subsection{对话}
{\kr 두 사람은 \ruby{韓食}{한식}집에 들어갔다.\\}
{\kr \begin{tabular}{lll}
    \ruby{金美善}{김미선}&:& 덥지요? \\
    \ruby{죤슨}{Johnson}&:& 예, 덥습니다. \\
    \ruby{金美善}{김미선}&:& 여보세요 여기 물 좀 주십시오.\\
    \ruby{죤슨}{Johnson}&:& 뭘 잡수시겠습니까?\\
    \ruby{金美善}{김미선}&:& 저는 \ruby{冷麵}{냉면}을 먹겠습니다.\\
    \ruby{죤슨}{Johnson}&:& 그럼, 불고기하고 \ruby{冷麵}{냉면}을 시킵시다.\\
\end{tabular}\\}
\textit{单词表} \\
\begin{tabular}{llllll}
    덥다&热&물&水&주다&给\\
    잡수시다&吃 (敬语)&\ruby{冷麵}{냉면}&冷面&먹다&吃\\
    그럼&那么&불고기&烤肉&시키다&要(菜), 点(菜)
\end{tabular}\\
\subsection{对话}
{\kr \begin{tabular}{lll}
    \ruby{金美善}{김미선}&:& 불고기 맛이 어떻습니까?\\
    \ruby{죤슨}{Johnson}&:& 참 맛이 있습니다.\\
    \ruby{金美善}{김미선}&:& 이것 좀 잡수십시오.\\
    \ruby{죤슨}{Johnson}&:& 그것이 무엇입니까?\\
    \ruby{金美善}{김미선}&:& 오이김치입니다.\\
    \ruby{죤슨}{Johnson}&:& 맵습니까?\\
    \ruby{金美善}{김미선}&:& 아니요, 맵지 않습니다.\\
\end{tabular}\\}
\textit{单词表} \\
\begin{tabular}{llllllll}
    맛&味道&참&真&오이&黄瓜&맵다&辣\\
\end{tabular}\\
\subsection{对话}
{\kr \begin{tabular}{lll}
    \ruby{金美善}{김미선}&:& 불고기를 더 시킬까요?\\
    \ruby{죤슨}{Johnson}&:& 아니요,많이 먹었습니다.\\
    \ruby{金美善}{김미선}&:& 그럼,\ruby{茶}{차}를 마십시다.\\
    \ruby{죤슨}{Johnson}&:& \ruby{人蔘茶}{인삼차}가 어떻습니까?\\
    \ruby{金美善}{김미선}&:& 저는 \ruby{커피}{coffee}를 마시겠습니다.\\
    \ruby{죤슨}{Johnson}&:& 여보세요, \ruby{人蔘茶}{인삼차}하고 \ruby{커피}{coffee}를 주십시오.\\
\end{tabular}\\}
\textit{单词表} \\
\begin{tabular}{llllllll}
    더&再&\ruby{茶}{차}&茶&마시다&喝&\ruby{人蔘茶}{인삼차}&人参茶\\
\end{tabular}\\
\subsection{课文}
나는 한 \ruby{時}{시}에 \ruby{點心}{점심}을 먹습니다.\\\indent
나는 \ruby{韓國}{한국} \ruby{飲食}{음식}을 좋아합니다.\\\indent
오늘은 \ruby{親舊}{친구}와 같이  \ruby{食堂}{식당}에 갔습니다.\\\indent
비빔밥하고 \ruby{冷麵}{냉면}을 시켰습니다.\\\indent
그 집  \ruby{飲食}{음식}이 참 맛이 있었습니다.\\\indent
값도 비싸지 않았습니다.\\\indent
아가씨도 \ruby{親切}{친절}했습니다.\\
\textit{单词表} \\
\begin{tabular}{llllll}
    \ruby{點心}{점심}&午饭&오늘&今天&값&价格,价钱\\
    비싸다&贵&아가씨&小姐&\ruby{親切}{친절}하다&热情,亲切
\end{tabular}\\
\section{\kr \ruby{文法}{문법}}
\begin{grammar}
    \begin{grammarsect}[韩国语的固有数词\\]
    \begin{tabular}{rlrlrlrlrlrlrlrlrlrlrl}
        1&하나&2&둘&3&셋&4&넷&5&다섯&6&여섯&7&일곱&8&여넓&9&아홉\\
        10&열&11&열하나&12&열둘\\
        20&스물&30&서른&40&마혼&50&쉰&60&예순&70&일혼&80&여든&90&아혼&100&\ruby{百}{백}
    \end{tabular}\\
    \begin{itemize}
        \item 韩国语的部分固有数词与量词(包括傲量词用的名词)连用时要产生音变现象, 这时하나、둘、셋、넷要变成한、두、세、네。
    \end{itemize}
    \begin{tabular}{lllllll}
        \ruby{例}{예}: &한 \ruby{時}{시}&두 \ruby{時}{시}&세 \ruby{時}{시}&네 \ruby{時}{시}&다섯 \ruby{時}{시}&여섯 \ruby{時}{시}\\
        &일곱 \ruby{時}{시}&여덟 \ruby{時}{시}&아홉 \ruby{時}{시}&열 \ruby{時}{시}&열한 \ruby{時}{시}&열두 \ruby{時}{시}\\
    \end{tabular}\\
    \end{grammarsect}
\end{grammar}
\begin{grammar}
    \begin{grammarsect}[-겠-]
    \begin{itemize}
        \item 时制词尾。表示未来时, 用于谓词词干和体词谓词形后,根据主语人称的不同,可以表示意志、推测、可能等。
        \item {\color{gray} Ⅰ-겠-}
    \end{itemize}
    \begin{tabular}{lll}
        \ruby{例}{예}: &오늘 저는 집에 있겠습니다. &今天我要待在家里。\\
        &\ruby{來日}{내일} 다시 오겠습니다.&我明天再来。\\
        &저녁에는 \ruby{親舊}{친구}를 만나겠습니다.&我晚上要见朋友。\\
        &오늘 밤에 \ruby{電話}{전화}하겠습니다.&我今天夜里打电话。
    \end{tabular}\\
    \end{grammarsect}
    \begin{grammarsect}[-지요]
        \begin{itemize}
            \item 准敬阶终结词尾。用于谓词词干和体词谓词形后。当用于疑问句时,表示说话人对所提问题已有所知,只是希望得到对方的确认,类似于汉语的“……吧?”。
            \item 在口语中,由句尾语调的升降,决定句子表示陈述、疑问、命令、共动等句式。
            \item {\color{gray} Ⅰ-지-요。Ⅰ-지形成半语终结词尾,加入요成为准敬阶终结词尾。有时지요缩写成죠。} 
        \end{itemize}
        \begin{tabular}{lll}
            \ruby{例}{예}: &그 일은 제가 하지요.&那事我做吧。\\
            &제가 돈을 내지요.&我来付钱吧。 \\
            &날씨가 덥지요?&天气热吧? \\
            &\ruby{先生}{선생}님이 좋지요?&老师很好吧? \\
            &먼저 가시지요.&请先走吧。\\
            &더 잡수시지요.&请再吃点儿吧。\\
            &같이 가시지요.&一起去吧。\\
            &\ruby{來日}{내일} 또 만나지요.&明天再见吧。\\
        \end{tabular}\\
    \end{grammarsect}
\end{grammar}
\begin{grammar}
    \begin{grammarsect}[-지 않다]
    \begin{itemize}
        \item 惯用型。用于谓词词干后,表示否定。不能用于命令句和共动句。
        \item {\color{gray} Ⅰ-지 않다.}
    \end{itemize}
    \begin{tabular}{lll}
        \ruby{例}{예}: &\ruby{來日}{내일}은 \ruby{學校}{학교}에 가지 않습니다. &明天不去学校。\\
        &그 \ruby{學生}{학생}은 \ruby{宿題}{숙제}를 하지 않습니다.& 那学生不傲作业。\\
        &요즘은 바쁘지 않습니까? &最近不忙吗?\\
        &\ruby{氣分}{기분}이 좋지 않습니다. &情绪不好。\\
    \end{tabular}\\
    \end{grammarsect}
\end{grammar}
\begin{grammar}
    \begin{grammarsect}[]
    \begin{itemize}
        \item 
    \end{itemize}
    \end{grammarsect}
\end{grammar}
\begin{grammar}
    \begin{grammarsect}[-에]
    \begin{itemize}
        \item 
    \end{itemize}
    \end{grammarsect}
    \begin{grammarsect}[-과 / -와]\label{gram:ua}
        
    \end{grammarsect}
\end{grammar}