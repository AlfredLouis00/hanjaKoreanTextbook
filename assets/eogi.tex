\chapter{语基}\label{chap:eogi}
\section{语基}
%\begin{table}[htbp]
%    \centering
%    \caption{语基}
\noindent\begin{tabular}{|c|c|c|c|c|c|c|}
	\hline
	                 & \multicolumn{3}{c|}{规则变形}          & \multicolumn{3}{c|}{特殊变形}                                                           \\\hline
	\diagbox{语基}{词干} & 元音词干                               & {\kr ㄹ}词干                            & 收音词干                         & 元音词干   & {\kr ㄹ}词干 & 收音词干 \\\hline
	第Ⅰ语基             & \multicolumn{3}{c|}{原形去除结尾的{\kr 다}(即词干)} & \multicolumn{3}{c|}{原形去除结尾的\kr 다}                                                       \\\hline
	第Ⅱ语基             & \multicolumn{2}{c|}{与第Ⅰ语基一样}       & 加\kr 으                             & \multicolumn{2}{c|}{与第Ⅰ语基一样} & 不规则地加\kr 으              \\\hline
	第Ⅲ语基             & \multicolumn{3}{c|}{加{\kr 아}或者{\kr 어}}         & \multicolumn{3}{c|}{不规则地加{\kr 아}或者{\kr 어}}                                                      \\\hline
\end{tabular}\\
%    \label{label}
%\end{table}
\indent 语基是一套由日本人提出的韩语语法体系\footnote{在本书中“语基”可能即表示这套语法体系也表示具体的词的基础形式,但是后一种情况时一般是以“第几语基”的形式出现,请读者注意区分},主要用于解释韩语的各种词的变形以及其连接的形式。目前市面上
%\footnote{这个语法术语是从日语借来的。用言是指需要变形的词,
%包括形容词动词等,和体言相对。即体言是不变形的词(即原书所谓体词),比如名词。
%接续是指用言后面连接的关系}。
\\\indent 简单来说,语基是指需要变形的词的基本形式,从上图可以看出语基体系里面共有三种语基,每种语基后的连接关系是固定的,比如Ⅲ-\footnote{``-''在{\kr 요}前表示在{\kr 요}前可以加东西,如果是{\kr -ㅆ-}则表示{\kr ㅆ}前后都可以加东西}{\kr 요}表示第Ⅲ语基后面加上{\kr 요},是非格式体敬语。但是这样就要记住各种接续所要求的语基类型。学习语基更应该考虑收音的放置,要自动把形如“-单独辅音”的谚文视作前一个的收音。
\\\indent 下面我们来对语基的变化情况一一说明。
\subsection{第Ⅰ语基}
第Ⅰ语基的变化非常简单,基本上去除结尾的{\kr 다}就是第Ⅰ语基了。但是{\kr ㄹ}词干的词有一点点特殊,后面连接不同的词会影响其第Ⅰ和第Ⅱ语基的形态,即{\kr ㄹ}的脱落。其实{\kr ㄹ}脱落就是遇见{\kr ㅅㅂ오ㄴ}脱落的音变规则(但是不绝对)。在本书中,{\kr ㄹ}脱落的情况我们标在语基的罗马数字后加上一个*,例如\ref{grm:bsio}的{\kr Ⅱ*-십시오}。
\subsection{第Ⅱ语基}
第Ⅱ语基的变化略复杂一点点。主要是收音词干的需要额外加一个{\kr 으},其余同第Ⅰ语基。然后注意有一些词是不规则变化。详情见\ref{eogi:bk}。
\subsection{第Ⅲ语基}
第Ⅲ语基就是统一在词干后面加上{\kr 아}或者{\kr 어}。如果词干最后一个元音是阳性元音{\kr ㅏㅑㅘㅗㅛ}之一,则加{\kr 아},否则加{\kr 어}。但是元音词干加{\kr 아}或者{\kr 어}会出现一些合并现象。\\
\begin{tabular}{|c|l|l|c|c|l|}
	\hline
	词干末元音              & 原形   & 词干 &\kr  아/어                & 第Ⅲ语基 & 说明                         \\\hline
	\multirow{2}{*}{\kr ㅏ} & {\kr 사다} 买 & \kr 사  & \kr 아                  & \kr 사    & {\kr ㅏ}和{\kr ㅏ}重复,合并                   \\\cline{2-6}
	                   & {\kr 하다} 做 & \kr 하  & -                  & \kr 해    & {\kr 하다}类词特殊变化,参照\ref{eogi:hata} \\\hline
	\multirow{2}{*}{\kr ㅗ} & {\kr 오다} 来 & \kr 오  & \multirow{2}{*}{\kr 아} & \kr 와    & {\kr ㅗ}与{\kr ㅏ}合为{\kr ㅘ}                     \\\cline{2-3}\cline{5-6}
	                   & {\kr 보다} 看 & \kr 보  &                    & \kr 봐/보아 & 前者常见但也有使用后者的               \\\hline
\end{tabular}\\
\begin{tabular}{|c|l|l|c|c|l|}
	\hline
	\multicolumn{6}{|c|}{接上表}                                                                                                                \\\hline
	词干末元音              & 原形                     & 词干                  &\kr  아/어                & 第Ⅲ语基                    & 说明                    \\\hline
	\multirow{2}{*}{\kr ㅓ} & {\kr 서다} 站立                  & \kr 서                   &\kr  어                  & \kr 서                       & {\kr ㅓ}和{\kr ㅓ}重复,合并              \\\cline{2-6}
	                   & {\kr 이러다} 这样做                & \kr 이러                  & -                  & \kr 이래                      & {\kr 어}特殊变化,参考\ref{eogi:eo} \\\hline
	\multirow{3}{*}{\kr ㅜ} & {\kr 배우다} 学习                 &\kr  배우                  & \multirow{2}{*}{\kr 어} & \kr 배워                      & {\kr ㅜ}和{\kr ㅓ}合为{\kr ㅝ}                \\\cline{2-3}\cline{5-6}
	                   & {\kr 주다} 做,请                 & \kr 주                   &                    & \kr 줘/주어                    & 也有使用{\kr 주어}不合并的            \\\cline{2-6}
	                   & {\kr 푸다} 汲取                  & \kr 푸                   & -                  & \kr 퍼                       & {\kr 우}特殊变化,参考\ref{eogi:u}  \\\hline
	\multirow{2}{*}{\kr ㅣ} & {\kr 치다} 打                   &\kr  치                   & \multirow{2}{*}{\kr 어} & \kr 쳐                       &{\kr  ㅣ}和{\kr ㅓ}合为\kr ㅕ                \\\cline{2-3}\cline{5-6}
	                   & {\kr 피다} (花)开                & \kr 피                   &                    & \kr 피어                      & 也有不合并成{\kr ㅕ}的              \\\hline
	\multirow{2}{*}{\kr ㅕ} & \multirow{2}{*}{{\kr 켜다} 点}  & \multirow{2}{*}{\kr 켜}  & \multirow{2}{*}{\kr 어} & \multirow{2}{*}{\kr 켜/켜어}   &{\kr  어}通常可以省略,              \\
	                   &                        &                     &                    &                         & 但也有加上{\kr 어}的用法             \\\hline
	\multirow{2}{*}{\kr ㅐ} & \multirow{2}{*}{{\kr 지내다} 过} & \multirow{2}{*}{\kr 지내} & \multirow{2}{*}{\kr 어} & \multirow{2}{*}{\kr 지내/지내어} & {\kr  어}通常可以省略,              \\
	                   &                        &                     &                    &                         & 但也有加上{\kr 어}的用法             \\\hline
	\multirow{2}{*}{\kr ㅔ} & \multirow{2}{*}{{\kr 베다} 割}  & \multirow{2}{*}{\kr 베}  & \multirow{2}{*}{\kr 어} & \multirow{2}{*}{\kr 베/베어}   & {\kr  어}通常可以省略,              \\
	                   &                        &                     &                    &                         & 但也有加上{\kr 어}的用法             \\\hline
	\multirow{2}{*}{\kr ㅚ} & \multirow{2}{*}{{\kr 되다} 变成} & \multirow{2}{*}{\kr 되}  & \multirow{2}{*}{\kr 어} & \multirow{2}{*}{\kr 돼/되어}   & {\kr ㅚ}和{\kr ㅓ}合为\kr ㅙ,               \\
	                   &                        &                     &                    &                         & 也有只是加上{\kr 어}的用法            \\\hline
	\kr ㅟ                  & {\kr 뛰다} 跳                   & \kr 뛰                   &\kr  어                  &\kr  뛰어                      & 单纯加\kr 어                  \\\hline
	\kr 一                  & {\kr 쓰다} 写                   & \kr 쓰                   & \kr 어                  & \kr 써                       & {\kr 으}特殊变化,参考\ref{eogi:eu} \\\hline
\end{tabular}\\
{\kr 이다}和{\kr 아니다}也有特殊的一些变化如下。\\
\begin{tabular}{|c|c|c|c|c|l|}
	\hline
	词干末元音              & 原形                   & 词干                  & \kr 아/어                & 第Ⅲ语基               & 说明              \\\hline
	\multirow{7}{*}{\kr ㅣ} & \multirow{5}{*}{\kr -이다} & \multirow{5}{*}{\kr 이}  & \multirow{5}{*}{\kr 어} & \kr 이어                 & 一般是直接加{\kr 어}成{\kr 이어}      \\\cline{5-6}
	                   &                      &                     &                    & \multirow{2}{*}{\kr 여} & 后接表示过去式的Ⅲ{\kr -ㅆ-}时, \\
	                   &                      &                     &                    &                    & {\kr -이다}前是元音词干的词则用\kr 여  \\\cline{5-6}
	                   &                      &                     &                    & \kr 이에                 & 后接终止形Ⅲ{\kr  -요}时,{\kr -이다}前 \\\cline{5-5}
	                   &                      &                     &                    & \kr 예                  & 没有收音则{\kr 이에},有收音则{\kr 예}。  \\\cline{2-6}
	                   & \multirow{2}{*}{\kr 아니다} & \multirow{2}{*}{\kr 아니} & \multirow{2}{*}{\kr 어} & \kr 아니어                & 通常直接加\kr 어          \\\cline{5-6}
	                   &                      &                     &                    & \kr 아니에                & 后接终止形Ⅲ {\kr -요}时变\kr 아니에。 \\\hline
\end{tabular}\\
\section{特殊变化}\label{eogi:bk}
\subsection{仅第Ⅲ语基不规则的特殊变化(元音词干)}
\subsubsection{{\kr 하다}特殊变化}\label{eogi:hata}
第Ⅲ语基变为{\kr 해}。也可以变作{\kr 하여},但{\kr 하여}是书面语,会有一种生硬的感觉。一般使用{\kr 해}就够了。
\subsubsection{{\kr 으}类词特殊变化}\label{eogi:eu}
词干以{\kr 으}结尾的某些词,第Ⅲ语基要将结尾的一替换为{\kr ㅏ}或者{\kr ㅓ}。词干有两个音节以上的,与第Ⅲ语基正常变化的类似,如果词干倒数第二个元音是阳性元音{\kr ㅏㅑㅘㅗㅛ}之一,则变{\kr ㅏ},否则变{\kr ㅓ}。词干是单音节的,直接变为{\kr ㅓ}。{\color{red} 注意,不是所有词干以{\kr 으}结尾的动词都是{\kr 으}类词特殊变化,还可能是{\kr 르}类词特殊变化和{\kr 러}类词特殊变化}
\subsubsection{{\kr 르}类词特殊变化}%\label{eogi:leu}
词干以{\kr 르}结尾的某些词,第Ⅲ语基要将结尾的{\kr 르}替换为{\kr ㄹ라}或者{\kr ㄹ러}。这也类似第Ⅲ语基正常变化。如果词干倒数第二个元音(不存在{\kr 르다}这个词)是阳性元音{\kr ㅏㅑㅘㅗㅛ}之一,则变{\kr ㄹ라},否则变{\kr ㄹ러}。
\subsubsection{{\kr 러}类词特殊变化}%\label{eogi:leo}
词干以{\kr 르}结尾的某些词,第Ⅲ语基要将结尾的{\kr 르}替换直接为{\kr 러}。
\subsubsection{{\kr 우}类词特殊变化}\label{eogi:u}
仅{\kr 푸다}一词第Ⅲ语基需要将{\kr ㅜ}变为{\kr ㅓ},即{\kr 퍼}。
\subsubsection{{\kr 어}类词特殊变化}\label{eogi:eo}
{\kr 이러다}、{\kr 그러다}、{\kr 저러다}、{\kr 어쩌다}四个词第Ⅲ语基需要将结尾的元音{\kr ㅓ}改成{\kr ㅐ}。
\subsection{第Ⅱ、Ⅲ语基都不规则变化的特殊变化(收音词干)}
\subsubsection{{\kr ㅂ}类词特殊变化}
部分词干收音是{\kr ㅂ}的词第Ⅱ语基变{\kr ㅂ}为{\kr 우},第Ⅲ语基变{\kr ㅂ}为{\kr 워}\footnote{除了{\kr 돕다}和{\kr 곱다}在第Ⅲ语基变{\kr ㅂ}为{\kr 와}}。
\\\indent 词干收音是{\kr ㅂ}的大部分形容词(排除{\kr 굽다、좁다}和{\kr 수줍다}等)和小部分动词(包括{\kr 돕다、굽다、눕다}和{\kr 줍다}等)是{\kr ㅂ}类词特殊变化
\subsubsection{{\kr ㄷ}类词特殊变化}
部分词干收音是{\kr ㄷ}的词第Ⅱ语基和第Ⅲ语基正常按规则变化分别加{\kr 으}和{\kr 아}或{\kr 어},同时将{\kr ㄷ}变为{\kr ㄹ}(这将使得除了词干收音是{\kr ㄷ}和{\kr ㄹ}从而不同的单词第Ⅱ语基和第Ⅲ语基相同)
\\\indent {\kr ㄷ}类词特殊变化的词不多。
\subsubsection{{\kr ㅎ}类词特殊变化}
部分词干收音是{\kr ㅎ}第Ⅱ语基{\kr ㅎ}脱落,第Ⅲ语基{\kr ㅎ}脱落的同时将词干结尾的元音变为{\kr ㅐ}。
\\\indent {\kr ㅎ}类词特殊变化只涉及形容词,动词没有{\kr ㅎ}类词特殊变化的。
\subsubsection{{\kr ㅅ}类词特殊变化}
部分词干收音是{\kr ㅅ}的词第Ⅱ语基和第Ⅲ语基正常按规则变化分别加{\kr 으}和{\kr 아}或{\kr 어},同时{\kr ㅅ}脱落。
\\\indent {\kr ㅅ}类词特殊变化的词不多。只包括{\kr 낫다、긋다、짓다、붓다、잇다}和{\kr 젓다}等。
