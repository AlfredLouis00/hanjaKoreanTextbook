\chapter{语基}\label{chap:eogi}
\section{语基}
\begin{table}[htbp]
    \centering
    \caption{语基}
        \begin{tabular}{|c|c|c|c|c|c|c|}
        \hline
        &\multicolumn{3}{c|}{规则变形}&\multicolumn{3}{c|}{特殊变形}\\\hline
        \diagbox{语基}{语干}&元音语干&ㄹ语干&辅音语干&元音语干&ㄹ语干&辅音语干\\\hline
        第Ⅰ语基&\multicolumn{3}{c|}{原形去除结尾的다(即语干)}&\multicolumn{3}{c|}{原形去除结尾的다}\\\hline
        第Ⅱ语基&\multicolumn{2}{c|}{与第Ⅰ语基一样}&加으&\multicolumn{2}{c|}{与第Ⅰ语基一样}&不规则地加으\\\hline
        第Ⅲ语基&\multicolumn{3}{c|}{加아或者어}&\multicolumn{3}{c|}{不规则地加아或者어}\\\hline
        \end{tabular}
    \label{label}
\end{table}
语基是一套由日本人提出的韩语语法体系,主要用于解释韩语的用言接续
\footnote{这个语法术语是从日语借来的。用言是指需要变形的词,
包括形容词动词等,和体言相对。即体言是不变形的词(即原书所谓体词),比如名词。
接续是指用言后面连接的关系}。目前市面上(包括日本)都没有一套使用语基进行语法解释的教科书。但是这样就要记住各种接续所要求的语基类型。
\\\indent 从上图可以看出语基体系里面共有三种语基,每种语基的接续关系是固定的。学习语基更应该考虑收音的放置,要自动把形如“-单独辅音”的谚文视作前一个的收音。
\\\indent 下面我们来对语基的变化情况一一说明。
\subsection{第Ⅰ语基}
第Ⅰ语基的变化非常简单,几乎就是去除结尾的다就是第Ⅰ语基了。但是ㄹ语干的词有一点点特殊,不同的接续会影响其第Ⅰ和第Ⅱ语基的形态,即ㄹ的脱落。但其实ㄹ脱落就是遇见ㅅㅂ오ㄴ脱落的音变规则(但是不绝对)。在本书中,ㄹ脱落的接续我们标在语基的罗马数字后加上一个*,例如第Ⅰ*语基。
\subsection{第Ⅱ语基}
第Ⅱ语基的变化略复杂一点点。主要是辅音词干的需要额外加一个으,其余同第Ⅰ语基。然后注意有一些词是不规则变化。详情见\ref{eogi:bk}。
\subsection{第Ⅲ语基}
第Ⅲ语基就是统一在语干后面加上아或者어。如果语干最后一个元音是阳性元音{\kr ㅏㅑㅘㅗㅛ}之一,则加아,否则加어。但是元音语干加아或者어会出现一些压缩现象。\\
\begin{tabular}{|c|l|l|l|c|l|}
    \hline
    语干末元音&原形&语干&아/어&第Ⅲ语基&说明\\\hline
    \multirow{2}{*}{ㅏ}&사다 买&사&아&사&ㅏ和ㅏ重复,合并\\\cline{2-6}
    &하다 做&하&-&해&하다用言特殊变化,参照\ref{eogi:hata}\\\hline
    \multirow{2}{*}{ㅗ}&오다 来&오&\multirow{2}{*}{아}&와&ㅗ与ㅏ合为ㅘ\\\cline{2-3}\cline{5-6}
    &보다 看&보&&봐/보아&前者常见但也有使用后者的\\\hline
    \multirow{2}{*}{ㅓ}&서다 站立&서&어&서&ㅓ和ㅓ重复,合并\\\cline{2-6}
    &이러다 这样做&이러&-&이래&어特殊变化,参考\ref{eogi:eo}\\\hline
\end{tabular}\\
\begin{tabular}{|c|l|l|l|c|l|}
    \hline
    \multicolumn{6}{|c|}{接上表}\\\hline
    语干末元音&原形&语干&아/어&第Ⅲ语基&说明\\\hline
    \multirow{3}{*}{ㅜ}&배우다 学习&배우&\multirow{2}{*}{어}&배워&ㅜ和ㅓ合为ㅝ\\\cline{2-3}\cline{5-6}
    &주다 做,请&주&&줘/주어&也有使用주어不合并的\\\cline{2-6}
    &푸다 汲取&푸&-&퍼&우特殊变化,参考\ref{eogi:u}\\\hline
    \multirow{2}{*}{ㅣ}&치다 打&치&\multirow{2}{*}{어}&쳐&ㅣ和ㅓ合为ㅕ\\\cline{2-3}\cline{5-6}
    &피다 (花)开&피&&피어&也有不合并成ㅕ的\\\hline
    \multirow{2}{*}{ㅕ}&\multirow{2}{*}{켜다 点}&\multirow{2}{*}{켜}&\multirow{2}{*}{어}&\multirow{2}{*}{켜/켜어}&어通常可以省略,\\
    &&&&&但也有加上어的用法\\\hline
    \multirow{2}{*}{ㅐ}&\multirow{2}{*}{지내다 过}&\multirow{2}{*}{지내}&\multirow{2}{*}{어}&\multirow{2}{*}{지내/지내어}&어通常可以省略,\\
    &&&&&但也有加上어的用法\\\hline
    \multirow{2}{*}{ㅔ}&\multirow{2}{*}{베다 割}&\multirow{2}{*}{베}&\multirow{2}{*}{어}&\multirow{2}{*}{베/베어}&어通常可以省略,\\
    &&&&&但也有加上어的用法\\\hline
    \multirow{2}{*}{ㅚ}&\multirow{2}{*}{되다 变成}&\multirow{2}{*}{되}&\multirow{2}{*}{어}&\multirow{2}{*}{돼/되어}&ㅚ和ㅓ合为ㅙ,\\
    &&&&&也有只是加上어的用法\\\hline
    ㅟ&뛰다 跳&뛰&어&뛰어&单纯加어\\\hline
    ㅡ&쓰다 写&쓰&어&써&으特殊变化,参考\ref{eogi:eu}\\\hline
\end{tabular}\\
이다和아니다也有特殊的一些变化如下。\\
\begin{tabular}{|c|c|c|c|c|l|}
    \hline
    语干末元音&原形&语干&아/어&第Ⅲ语基&说明\\\hline
    \multirow{7}{*}{ㅣ}&\multirow{5}{*}{-이다}&\multirow{5}{*}{이}&\multirow{5}{*}{어}&이어&一般是直接加어成이어\\\cline{5-6}
    &&&&\multirow{2}{*}{여}&后接表示过去式的Ⅲ -ㅆ-时,\\
    &&&&&-이다前是没有收音的体言则用여\\\cline{5-6}
    &&&&이에&后接终止形Ⅲ -요时,-이다前\\\cline{5-5}
    &&&&예&没有收音则이에,有收音则예。\\\cline{2-6}
    &\multirow{2}{*}{아니다}&\multirow{2}{*}{아니}&\multirow{2}{*}{어}&아니어&通常直接加어\\\cline{5-6}
    &&&&아니에&后接终止形Ⅲ -요时变아니에。\\\hline
\end{tabular}\\
\section{特殊变化}\label{eogi:bk}
\subsection{仅第Ⅲ语基不规则的特殊变化(元音语干)}
\paragraph{하다用言特殊变化}\label{eogi:hata}
第Ⅲ语基变为해。也可以变作하여,但하여是书面语,会有一种生硬的感觉。一般使用해就够了。
\paragraph{으用言特殊变化}\label{eogi:eu}
语干以으结尾的某些用言,第Ⅲ语基要将结尾的ㅡ替换为ㅏ或者ㅓ。语干有两个音节以上的,与第Ⅲ语基正常变化的类似,如果语干倒数第二个元音是阳性元音{\kr ㅏㅑㅘㅗㅛ}之一,则变ㅏ,否则变ㅓ。语干是单音节的,直接变为ㅓ。{\color{red} 注意,不是所有语干以으结尾的动词都是으用言特殊变化,还可能是르用言特殊变化和러用言特殊变化} 
\paragraph{르用言特殊变化}%\label{eogi:leu}
语干以르结尾的某些用言,第Ⅲ语基要将结尾的르替换为ㄹ라或者ㄹ러,也类似第Ⅲ语基正常变化。如果语干倒数第二个元音(不存在르다这个用言)是阳性元音{\kr ㅏㅑㅘㅗㅛ}之一,则变ㄹ라,否则变ㄹ러。
\paragraph{러用言特殊变化}%\label{eogi:leo}
语干以르结尾的某些用言,第Ⅲ语基要将结尾的르替换直接为러。
\paragraph{우用言特殊变化}\label{eogi:u}
仅푸다一词第Ⅲ语基需要将ㅜ变为ㅓ,即퍼。
\paragraph{어用言特殊变化}\label{eogi:eo}
이러다・그러다・저러다・어쩌다四个词第Ⅲ语基需要将结尾的元音ㅓ改成ㅐ。
\subsection{第Ⅱ、Ⅲ语基都不规则变化的特殊变化(辅音语干)}
\paragraph{ㅂ用言特殊变化}
部分语干收音是ㅂ的用言第Ⅱ语基变ㅂ为우,第Ⅲ语基变ㅂ为워\footnote{除了돕다和곱다在第Ⅲ语基变ㅂ为와}。
\\语干收音是ㅂ的大部分形容词(排除굽다、좁다和수줍다等)和小部分动词(包括돕다、굽다、눕다和줍다等)是ㅂ用言特殊变化
\paragraph{ㄷ用言特殊变化}
部分语干收音是ㄷ的用言第Ⅱ语基和第Ⅲ语基正常按规则变化分别加으和아或어,同时将ㄷ变为ㄹ(这将使得除了语干收音是ㄷ和ㄹ从而不同的单词第Ⅱ语基和第Ⅲ语基相同)
\\ㄷ用言特殊变化的词不多。
\paragraph{ㅎ用言特殊变化}
部分语干收音是ㅎ第Ⅱ语基ㅎ脱落,第Ⅲ语基ㅎ脱落的同时将语干结尾的元音变为ㅐ。
\\ㅎ用言特殊变化只涉及形容词,动词没有ㅎ用言特殊变化的。
\paragraph{ㅅ用言特殊变化}
部分语干收音是ㅅ的用言第Ⅱ语基和第Ⅲ语基正常按规则变化分别加으和아或어,同时ㅅ脱落。
\\ㅅ用言特殊变化的词不多。只包括낫다