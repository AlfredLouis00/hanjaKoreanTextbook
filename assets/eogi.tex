\chapter{语基}
\section{语基}
\begin{table}[htbp]
    \centering
    \caption{语基}
        \begin{tabular}{|c|c|c|c|c|c|c|}
        \hline
        &\multicolumn{3}{c|}{规则变形}&\multicolumn{3}{c|}{特殊变形}\\\hline
        \diagbox{语基}{语干}&元音语干&ㄹ语干&辅音语干&元音语干&ㄹ语干&辅音语干\\\hline
        第Ⅰ语基&\multicolumn{3}{c|}{原形去除结尾的다(即语干)}&\multicolumn{3}{c|}{原形去除结尾的다}\\\hline
        第Ⅱ语基&\multicolumn{2}{c|}{与第Ⅰ语基一样}&加으&\multicolumn{2}{c|}{与第Ⅰ语基一样}&不规则地加으\\\hline
        第Ⅲ语基&\multicolumn{3}{c|}{加아或者어}&\multicolumn{3}{c|}{不规则地加아或者어}\\\hline
        \end{tabular}
    \label{label}
\end{table}
语基是一套由日本人提出的韩语语法体系,主要用于解释韩语的用言接续
\footnote{这个语法术语是从日语借来的。用言是指需要变形的词,
包括形容词动词等,和体言相对。即体言是不变形的词(即原书所谓体词),比如名词。
接续是指用言后面连接的关系}。目前市面上(包括日本)都没有一套使用语基进行语法解释的教科书。但是这样就要记住各种接续所要求的语基类型。
\\\indent 从上图可以看出语基体系里面共有三种语基,每种语基的接续关系是固定的。学习语基更应该考虑收音的放置,要自动把形如“-单独辅音”的谚文视作前一个的收音。
\\\indent 下面我们来对语基的变化情况一一说明。
\subsection{第Ⅰ语基}
第Ⅰ语基的变化非常简单,几乎就是去除结尾的다就是第Ⅰ语基了。但是ㄹ语干的词有一点点特殊,不同的接续会影响其第Ⅰ和第Ⅱ语基的形态。在本书中,ㄹ脱落的接续我们标在语基的罗马数字后加上一个*,例如第Ⅰ*语基。
\subsection{第Ⅱ语基}
第Ⅱ语基的变化略复杂一点点。主要是辅音词干的需要额外加一个으,其余同第Ⅰ语基。然后注意有一些词是不规则变化。详情见\ref{eogi:bk}。
\subsection{第Ⅲ语基}
第Ⅲ语基就是统一在语干后面加上아或者어。如果语干最后一个元音是阳性元音{\kr ㅏㅑㅘㅗㅛ}之一,则加아,否则加어。但是元音语干加아或者어会出现一些压缩现象。\\
\begin{tabular}{|c|l|l|l|c|l|}
    \hline
    语干末元音&原形&语干&아/어&第Ⅲ语基&说明\\\hline
    \multirow{2}{*}{ㅏ}&사다 买&사&아&사&ㅏ和ㅏ重复,合并\\\cline{2-6}
    &하다 做&하&-&해&参照\ref{eogi:hada}\\\hline
    \multirow{2}{*}{ㅗ}&오다 来&오&\multirow{2}{*}{아}&와&ㅗ与ㅏ合为ㅘ\\\cline{2-3}\cline{5-6}
    &보다 看&보&&봐/보아&前者常见但也有使用后者的\\\hline
\end{tabular}\\